\documentclass{exam}


%%% Declare commands whether new or old
\newcommand{\declarecommand}[1]{\providecommand{#1}{}\renewcommand{#1}}

%------------------------------------------------------------------------
%	COURSE INFORMATION
%------------------------------------------------------------------------

%%% Global document information
\declarecommand{\university}{University of Toronto}
\declarecommand{\coursename}{Multivariable Calculus with Proofs}
\declarecommand{\coursecode}{MAT237}
\declarecommand{\term}{2022-23}
\declarecommand{\firstterm}{Fall 2022}
\declarecommand{\secondterm}{Winter 2023}

\declarecommand{\title}{\coursecode\, \coursename}
\declarecommand{\subtitle}{\docname}
\declarecommand{\subsubtitle}{Due \dueday\, \duedate \, by \duetime}
\declarecommand{\author}{S. Artamonov, F. Parsch, and A. Zaman}


%%% URLs
\declarecommand{\urlGrading}{https://www.gradescope.ca/courses/7779} % Gradescope
\declarecommand{\urlLMS}{https://q.utoronto.ca/courses/280409} % Quercus 
\declarecommand{\urlFAQ}{https://q.utoronto.ca/courses/280409/pages/policies-and-faq} % Quercus FAQ page


%%%%%%%%%%%%%%%%%%%%%%%%%%%%%%%%%%%%%%%%%%%%%%%%%%%%%%%%%%%%%%%%%%%%%%%%%%%%%%%%%%%%%%%%%%%%%%%%%%%%
%%%		PACKAGES
%%%%%%%%%%%%%%%%%%%%%%%%%%%%%%%%%%%%%%%%%%%%%%%%%%%%%%%%%%%%%%%%%%%%%%%%%%%%%%%%%%%%%%%%%%%%%%%%%%%%

%------------------------------------------------------------------------
%	FORMAT AND LAYOUT
%------------------------------------------------------------------------
\usepackage[24hr,iso]{datetime} % Required to write time of compilation
\usepackage[margin=1in]{geometry} % Required for setting margins
\parindent0pt % Set indentation to 0 pt
\usepackage[english]{babel} % English language/hyphenation

\declarecommand{\toggledraft}{
	\usepackage[firstpageonly=true]{draftwatermark}
	\def\answerborderwidth{1pt}
}

%------------------------------------------------------------------------
%	COLORS
%------------------------------------------------------------------------
\usepackage{xcolor} % required to defined colours
	\definecolor{DarkBlue}{HTML}{25355A}
	\definecolor{LightBlue}{HTML}{007FA3}
	\definecolor{Yellow}{HTML}{A37500}
	\definecolor{Red}{HTML}{A3002E}

%------------------------------------------------------------------------
%	FONTS
%------------------------------------------------------------------------
\usepackage[charter,cal=cmcal]{mathdesign} % Specify font for serif family
\usepackage{avant} % Specify avant font for sans serif family
\usepackage{mathtools} % Replaces amsmath
\usepackage{sectsty} % Specify section formatting
	\sectionfont{\sffamily\color{DarkBlue}}
	\subsectionfont{\sffamily\color{LightBlue}}
	\subsubsectionfont{\small\sffamily}

%------------------------------------------------------------------------
%	TABLES, IMAGES, AND FIGURES
%------------------------------------------------------------------------

\usepackage{array} % Required for tables
\usepackage{graphicx} % Required for including images
	\graphicspath{{img/}} % Specifies the directory where pictures are stored
\usepackage{tikz} % Required for creating plots
	\usetikzlibrary{decorations.pathreplacing}
\usepackage{tikz-3dplot} % Required for 3D plots
\usepackage{tikz-cd} % Required for commutative diagrams
\usepackage{pgfplots}
	\pgfplotsset{compat=1.12}
	\usepgfplotslibrary{colormaps}
	\usepgfplotslibrary{patchplots}
	\usepgfplotslibrary{fillbetween}

%------------------------------------------------------------------------
%	LINKS AND REFERENCES
%------------------------------------------------------------------------

\usepackage[hidelinks, urlcolor=Red, linkcolor=Yellow, colorlinks=true]{hyperref}   
\usepackage[noabbrev,capitalise,nameinlink]{cleveref}

%------------------------------------------------------------------------
%	BOXES, CHOICES, AND LISTS
%------------------------------------------------------------------------

\usepackage{enumerate}
\usepackage{enumitem} % Customize lists
% \setlist{nolistsep} % Reduce spacing between bullet points and numbered lists

\declarecommand\partlabel{(\thequestion\alph{partno})} % parts are labelled (1a)

\usepackage{environ}
%%% Answer environment makes a framed box of a chosen height
\NewEnviron{answer}[1]{
	\fullwidth{
	\vspace*{-10pt}
		\par\nobreak\vspace{\ht\strutbox}\noindent
		\setlength{\fboxrule}{\answerborderwidth}
		\fbox{% 
		\parbox[c][#1][t]{\dimexpr\linewidth-2\fboxsep}{
		\hrule width \hsize height 0pt \color{LightBlue} 
		\BODY
		}%
		}%
	}
}
\def\answerborderwidth{0pt} % Set answer boxes to be invisible by default

% Define remark environment
\NewEnviron{remark}{
	\fullwidth{
	\vspace*{-10pt}
		\color{Red} \textbf{Remark:}  
		\BODY
	}
}

\CorrectChoiceEmphasis{\color{LightBlue}\bf}	

\declarecommand{\CorrectChoiceBox}{
	\checkboxchar{$\Box$} 
	\checkedchar{$\blacksquare$}
}
\declarecommand{\CorrectChoiceCircle}{
	\checkboxchar{\tikz[baseline=-0.6ex]\draw[black,thick] (0,0) circle (1ex);}
	\checkedchar{\tikz[baseline=-0.6ex]\draw[LightBlue,fill=LightBlue,thick] (0,0) circle (1ex);}
}

 
%------------------------------------------------------------------------
%	THEOREMS
%------------------------------------------------------------------------

\usepackage{amsthm}

%%% Specify theorems
\newtheorem{theorem}{Theorem}
\newtheorem*{definition}{Definition}

\newtheoremstyle{named}{}{}{\itshape}{}{\bfseries}{.}{.5em}{\thmnote{#3}}
\theoremstyle{named}
\newtheorem*{namedtheorem}{Theorem}


%% Setup proof lines
\NewEnviron{lines}{
	\vspace*{-5pt}
	\begin{center}
	\fbox{\begin{minipage}{0.9\textwidth}
	\begin{enumerate}[label=\arabic*.]
		\it \BODY
	\end{enumerate}
	\end{minipage}
	}
	\end{center}
}
 
%------------------------------------------------------------------------
%	HEADER & FOOTER
%------------------------------------------------------------------------

%%% Define footnote without marker
\declarecommand{\blfootnote}[1]{%
  \begingroup
  \declarecommand\thefootnote{}\footnote{#1}%
  \addtocounter{footnote}{-1}%
  \endgroup
}

%%% Specify header and footer
\pagestyle{headandfoot} 	\runningfootrule  
	\runningfooter{\coursecode}{\docname\ - Page \the\numexpr\thepage\ of \numpages}{\duedate}

%------------------------------------------------------------------------
%	TITLE PAGE
%------------------------------------------------------------------------


%%% Define title
\declarecommand{\maketitle}{
	\begin{center}
		{\color{DarkBlue}\Large\sffamily\bfseries\title} \\[2pt]
		{\color{LightBlue}\Large\sffamily\bfseries\subtitle}
		\blfootnote{Created and revised by \author. Last updated \today \, at \currenttime.}\\
		{\sffamily\subsubtitle} 
	\end{center}
 
}

%%% Define title page
\declarecommand{\makeinstructions}
{ 
\section*{Instructions}
This problem set is based on \docinfo. Please read the \href{\urlFAQ}{Problem Set FAQ} for details on submission policies, collaboration  rules, and general instructions.  
\begin{itemize}
	\item \textbf{\href{\urlLMS/pages/tutorials}{Tutorials} on \tutorialdate\, will help you with \docname.} You will work with peers and get help from TAs. Before attending, seriously attempt these problems and prepare initial drafts.
	\item \textbf{Submissions are only accepted by \href{\urlGrading}{Gradescope}}. Do not send anything by email.  Late submissions are not accepted under any circumstance. Remember you can resubmit anytime before the deadline. 
	\item \textbf{Submit your polished solutions using only this template PDF.} You will submit a single PDF with your full written solutions. If your solution is not written using this template PDF (scanned print or digital) then you will receive zero. Do not submit rough work. Organize your work neatly in the space provided.  
	\item \textbf{Show your work and justify your steps} on every question, unless otherwise indicated. Put your final answer in the box provided, if necessary. 
\end{itemize}
We recommend you write draft solutions on separate pages and afterwards write your polished solutions here. You must fill out and sign the academic integrity statement below; otherwise, you will receive zero. 

\section*{Academic integrity statement}
%%% Student information
% Person A
\fbox{
\begin{minipage}{\textwidth}
{
	\vspace{0.2in}
	
	\makebox[\textwidth]{\sffamily Full Name:\enspace{\bfseries\large\FirstStudentName\,}\hrulefill}
	
	\vspace{0.2in}
	
	\makebox[\textwidth]{\sffamily Student number:\enspace{\bfseries\large\FirstStudentNumber\,}\hrulefill}
	
	\vspace{0.1in}
	
}
\end{minipage}
}

\vspace*{0.1in}

% Person B
\fbox{
\begin{minipage}{\textwidth}
{
\vspace{0.2in}

\makebox[\textwidth]{\sffamily Full Name:\enspace{\bfseries\large\SecondStudentName\,}\hrulefill}

\vspace{0.2in}

\makebox[\textwidth]{\sffamily Student number:\enspace{\bfseries\large\SecondStudentNumber\,}\hrulefill}

\vspace{0.1in}

}
\end{minipage}
}
~

I confirm that:

\begin{itemize} 
	\item I have read and followed the policies described in the \href{\urlFAQ}{Problem Set FAQ}. 
	\item I have read and understand the rules for collaboration on problem sets described in the Academic Integrity subsection of the syllabus. I have not violated these rules while writing this problem set. 
	\item I understand the consequences of violating the University's academic integrity policies as outlined in the \href{http://www.governingcouncil.utoronto.ca/policies/behaveac.htm}{Code of Behaviour on Academic Matters}. I have not violated them while writing this assessment.
\end{itemize}
By signing this document, I agree that the statements above are true. 

% You should sign this PDF after compiling. Do not write your signature using LaTeX.
\vspace{0.2in}
{\large 
	\makebox[\textwidth]{\sffamily Signatures: 1)\enspace\hrulefill} 
	
	\vspace{0.2in}
	
	\makebox[\textwidth]{\sffamily \hspace*{22mm} 2)\enspace\hrulefill} 
}
\vfill
\pagebreak
}



%%% Uncomment if you are writing a draft
%\toggledraft

%%% Uncomment if you want to show your multiple choice answers
\printanswers

%%% Local document information
\declarecommand{\docname}{Problem Set 7}		\usepackage{neuralnetwork}
\declarecommand{\dueday}{Friday}
\declarecommand{\duedate}{March 17, 2023}
\declarecommand{\duetime}{13:00 ET}
\declarecommand{\tutorialdate}{Tuesday March 14, 2023}
\declarecommand{\docinfo}{\href{\urlFAQ/modules}{Module J: Line integrals} (J1 to J5) and \href{\urlFAQ/modules}{Module K: Fundamental theorems in 2D} (K1)}

%%% Uncomment both if you are a MAT237 teaching member writing the official  solutions
%\declarecommand{\makeinstructions}{\vfill\pagebreak}
%\declarecommand{\docname}{Problem Set 7 solutions}
%\declarecommand{\authors}{C. Davies, A. Qiu, and A. Zaman}

%%% Student information
\declarecommand{\FirstStudentName}{} 
\declarecommand{\FirstStudentNumber}{}

\declarecommand{\SecondStudentName}{} % leave empty if you are submitting individually
\declarecommand{\SecondStudentNumber}{} % leave empty if you are submitting individually

%%% Custom math commands
%% Standard
\declarecommand{\ds}{\displaystyle} % force math styling
\declarecommand{\emptyset}{\varnothing} % better empty set
\declarecommand{\epsilon}{\varepsilon} % better epsilon
\declarecommand{\R}{\mathbb{R}} % reals
\declarecommand{\Z}{\mathbb{Z}} % integers
\declarecommand{\Q}{\mathbb{Q}} % rationals
\declarecommand{\N}{\mathbb{N}} % naturals
\declarecommand{\C}{\mathbb{C}} % complex
\DeclarePairedDelimiter\abs{\lvert}{\rvert} % absolue value
\DeclarePairedDelimiter\norm{\lVert}{\rVert} % norm

%% Linear algebra
\declarecommand{\mat}[1]{\begin{bmatrix*}[r]#1\end{bmatrix*}}
\declarecommand{\matc}[1]{\begin{bmatrix}#1\end{bmatrix}} % matrix
\DeclareMathOperator{\Span}{span} % span
\DeclareMathOperator{\Img}{img} % image
\DeclareMathOperator{\Ker}{ker} % kernel
\DeclareMathOperator{\Id}{id} % identity map
\DeclareMathOperator{\Range}{range} % range
\DeclareMathOperator{\Rref}{rref} % row reduced echelon form
\DeclareMathOperator{\Rank}{rank} % rank
\DeclareMathOperator{\Null}{null} % null space
\DeclareMathOperator{\Nullity}{nullity} % nullity
\DeclareMathOperator{\Proj}{proj} % projection
\DeclareMathOperator{\Dim}{dim} % dimension

%% Topology
\DeclareMathOperator{\Int}{int}
\DeclareMathOperator{\Cl}{cl}

%% Calculus
\DeclareMathOperator{\Length}{length} % length
\DeclareMathOperator{\Area}{area} % area
\DeclareMathOperator{\Vol}{vol} % volume
\DeclareMathOperator{\Grad}{grad} % gradient
\DeclareMathOperator{\Curl}{curl} % curl
\DeclareMathOperator{\Div}{div} % divergence

\usepackage{amsmath}

\begin{document}



\maketitle

\vspace*{-15pt}
\makeinstructions
 
%%% PROBLEMS
\section*{Problems} 

\CorrectChoiceCircle % for multiple choice questions

\begin{questions}


%%% QUESTION 1
\question  A 70 kilogram person slides down a frictionless slide from a point $30$ metres above ground to a point on the ground under the influence of the gravitational force $F(x,y,z) = (0, 0, -9.8 \cdot 70)$ measured in Newtons. The slide follows a spiral path which circles around the $z$-axis and, when viewed from above,  looks like a circle of radius $2$ metres. Starting $30$ metres up, the slide circles around five full times. ({\color{Yellow}Revised 2023-03-13})
\begin{parts}


\item Choose a parametrization of the path the person will follow starting at time $t=0$ at the top of the slide. It should be a simple regular parametrization, but you do not need to prove it. 

%% Do not change the height of this box; your work must fit inside
\begin{answer}{200pt}
% Insert your answer here
Define $C$ as the path of the slide. A picture of $C$ is shown to the right:\\\\\\\\\\\\

Suppose it takes 30 seconds to slide down $C$, so the height travelled is proportional to the time elapsed. Since the $z$ component should trace the height of the slide, it should be $t$. The $x$ and $y$ components should trace a circle of radius 2, so they should have a 2cos and 2sin term respectively. Since $C$ circles around the $z$-axis five times, the variable inside the 2cos and 2sin should be $\frac{\pi t}{3}$ since $30\frac{\pi}{3} = 10\pi$, or five rotations.\\

Thus, define $\gamma: [0, 30] \to \R^3$ by $\gamma(t) = (2\cos(\frac{\pi t}{3}), 2\sin(\frac{\pi t}{3}), t)$ for $t \in [0, 30]$. Notice it is a simple and regular parametrizataion of $C$.\\\\
\end{answer}

\item By calculating from definition, find the distance travelled (with units) by the person on the slide.  

%% Do not change the height of this box; your work must fit inside
\begin{answer}{300pt}
% Insert your answer here
The distance is the arc length. By definition 11.2.1, the arc length of $C$ is $\ell(C) = \int_{0}^{30}\norm{\gamma'(t)}dt = \int_{0}^{30}\norm{(-\frac{2\pi}{3}\sin(\frac{\pi t}{3}), \frac{2\pi}{3}\cos(\frac{\pi t}{3}), 1)}dt = \int_{0}^{30} \sqrt{(\frac{2\pi}{3})^2 + 1}dt = \int_{0}^{30} \frac{\sqrt{4\pi^2 + 9}}{3}dt = 10\sqrt{4\pi^2 + 9} \approx 69.63$. Thus, the person traveled $69.63$ meters on the slide.
\end{answer}


\item By calculating from definition, find the amount of work done (with units) on the person by $F$.
%% Do not change the height of this box; your work must fit inside
\begin{answer}{320pt}
% Insert your answer here
By definition 11.3.11, the work done by F on the person along $C$ is $\int_C F \cdot T ds = \int_0^{30} F(\gamma(t)) \cdot T(t) \norm{\gamma'(t)}dt = \int_0^{30} F(\gamma(t)) \cdot \gamma'(t)dt = \int_0^{30} (0,0,-9.8 \cdot 70) \cdot (-\frac{2\pi}{3}\sin(\frac{\pi t}{3}), \frac{2\pi}{3}\cos(\frac{\pi t}{3}), 1) dt = \int_0^{30} -9.8 \cdot 70 dt = 30 \cdot -9.8 \cdot 70 = -20580$. Thus, the work done is -20,580 Joules.
\end{answer}

\item If possible, use the fundamental theorem of line integrals to calculate the work done (with units) by $F$. If not, explain why not. 

%% Do not change the height of this box; your work must fit inside
\begin{answer}{250pt}
% Insert your answer here
Define $f: \R^3 \to \R$ by $f(x,y,z) = (-9.8\cdot70\cdot z)$. Notice $f$ is $C^1$. Since $F = \nabla f$, by the fundamental theorem of line integrals, $\int_C F \cdot d\gamma = \int_C \nabla f \cdot d\gamma = f(\gamma(30)) - f(\gamma(0)) = 30 \cdot -9.8 \cdot 70 = -20580$ Joules, the same answer as in b).
\end{answer}



\end{parts}



\pagebreak
%%% QUESTION 2
\question  Let $C$ be an oriented curve that is parametrized by   $\gamma_1 : [a,b] \to \R^n$ and also by $\gamma_2 : [c,d] \to \R^n$. 

Let $F$ be a vector field in $\R^n$ that is continuous on $C$. Prove  that 
\[
\ds \int_a^b F(\gamma_1(t)) \cdot \gamma_1'(t) dt = \int_c^d F(\gamma_2(t)) \cdot \gamma_2'(t) dt.
\]
(Hence, the line integral of $F$ along $C$ is well-defined.)

%% Do not change the height of this box; your work must fit inside
\begin{answer}{550pt}
% Insert your answer here
Since $\gamma_1$ and $\gamma_2$ parameterize the same oriented curve, they are reparametrizations
of each other with the same orientation. By definition 11.1.24, this means there exists a continuous invertible $\phi: [a,b] \to [c,d]$ such that $\phi$ is $C^1$ on (a,b) with $\phi'>0$ and $\gamma_1 = \gamma_2 \circ \phi$.\\

Then, $F(\gamma_1(t))\cdot\gamma_1'(t) = F(\gamma_2(\phi(t)))\cdot(\gamma_2'(\phi(t))\phi'(t))$ since $\gamma_1'(t) = \gamma_2'(\phi(t))\phi'(t)$ by the chain rule.\\

Since $\phi'>0$ on $(a,b)$ and $\phi$ is continuous and invertible on [a, b], $\phi(a)=c$ and $\phi(b)=d$. Since $F$, $\gamma_2$, $\phi$, and $\phi'$ are continuous on their respective domains, $(F(\gamma_2(\phi(t)))\cdot\gamma_2'(\phi(t)))\phi'(t)$ is integrable on $t \in [a,b]$. By 1D substitution of integrals, this means $(F\circ\gamma_2)\cdot\gamma_2'$ is integrable on $[c,d]$.\\

Thus, using the substitution $u = \phi(t)$ and $du = \phi'(t)dt$, we get $\int_a^b F(\gamma_1(t))\cdot\gamma_1'(t)dt = \int_a^b F(\gamma_2(\phi(t)))\cdot(\gamma_2'(\phi(t))\phi'(t))dt = \int_a^b (F(\gamma_2(\phi(t)))\cdot\gamma_2'(\phi(t)))\phi'(t)dt = \int_{\phi(a)}^{\phi(b)} F(\gamma_2(u))\cdot\gamma_2'(u)du = \int_{c}^{d} F(\gamma_2(u))\cdot\gamma_2'(u)du = \int_{c}^{d} F(\gamma_2(t))\cdot\gamma_2'(t)dt$ as needed.
\end{answer}



\pagebreak
%%% QUESTION 3
\question ({\color{Yellow} Revised 2023-03-13}) Consider the following true theorem.
\begin{quote}
	\textbf{Theorem A.} \it Let $U \subseteq \R^n$ be a non-empty $C^1$ path-connected open set. Let $F$ be a vector field in $\R^n$ that is continuous on $U$. If $\ds \int_C F \cdot d\gamma = 0$ for any closed piecewise curve $C$ lying in $U$, then there exists a $C^1$ function $f : U \to \R$ such that $F = \nabla f$. 
\end{quote}
Here is a WRONG proof of Theorem A.   
\begin{lines}
	\item Fix $a \in U$. For each $x \in U$, choose a curve $C_x$ from $a$ to $x$ lying inside $U$. 
	\item Define $f : U \to \R$ by $f(x) = \int_{C_{x}} F \cdot d\gamma$ for $x \in U$. 
	\item Fix $j \in \{1,\dots,n\}$ and let $\{e_1,\dots,e_n\}$ be the standard basis in $\R^n$. 
	\item Let $\epsilon > 0$ be such that $B_{\epsilon}(a) \subseteq U$. 
	\item For $h \in (-\epsilon,\epsilon)$ with $h \neq 0$, define $L_{a+he_j}$ to be the straight line segment from $a$ to $a+he_j$, so that
		\[
		\frac{f(a+he_j)-f(a)}{h} = \frac{1}{h}\int_{C_{a+he_j}} F \cdot d\gamma = \frac{1}{h} \int_{L_{a+he_j}} F \cdot d\gamma . 	
		\]
	\item Defining $\gamma : [0,1] \to \R^n$ by $\gamma(t) = a + the_j$, it follows that
		\[
		\frac{1}{h} \int_{L_{a+he_j}} F \cdot d\gamma = \frac{1}{h} \int_0^1 F(a+the_j) \cdot (h e_j) dt =  \int_0^1 F_j(a+the_j) dt = \frac{1}{h} \int_0^h F_j(a+te_j) dt
		\]
	\item By the fundamental theorem of calculus, $\ds \partial_j f(a) =  \lim_{h \to 0} \left[ \frac{1}{h} \int_0^h F_j(a+te_j) dt \right] = F_j(a).$ 
	\item Since $a$ and $j$ were arbitrary, this proves that $F = \nabla f$. 
\end{lines}
You will identify when specific assumptions are required, and you will also identify the fatal error. 

\begin{parts}

\item Which line(s) require that $U$ is open? 

\CorrectChoiceBox
% Replace \choice with \CorrectChoice to select your desired answer(s)
\begin{oneparcheckboxes}
	\choice Line 1
	\choice Line 2
	\choice Line 3
	\CorrectChoice Line 4
	\choice Line 5		
	\choice Line 6		
	\choice Line 7		
	\choice Line 8							
\end{oneparcheckboxes}
\medskip

\item Which line(s) require that $U$ is path-connected? 

\CorrectChoiceBox
% Replace \choice with \CorrectChoice to select your desired answer(s)	
\begin{oneparcheckboxes}
	\CorrectChoice Line 1
	\choice Line 2
	\choice Line 3
	\choice Line 4
	\choice Line 5		
	\choice Line 6		
	\choice Line 7		
	\choice Line 8							
\end{oneparcheckboxes}
\medskip

\item Which line(s) require that the integral of $F$ along any piecewise closed curve is zero?

\CorrectChoiceBox
% Replace \choice with \CorrectChoice to select your desired answer(s)
\begin{oneparcheckboxes}
	\choice Line 1
	\choice Line 2
	\choice Line 3
	\choice Line 4
	\CorrectChoice Line 5		
	\choice Line 6		
	\choice Line 7		
	\choice Line 8							
\end{oneparcheckboxes}
\medskip

\item Which line has the false claim in this argument? Identify the line and describe the flaw in  $\leq 100$ words. 

%% Do not change the height of this box; your work must fit inside
\begin{answer}{130pt}
% Insert your answer here
The false line is line 8. Instead of showing $\exists f$ such that $F(a) = \nabla f(a)$, $\forall a \in U$, the proof shows $\forall a \in U$, $\exists f$ such that $F(a) = \nabla f(a)$. This is because in the proof, the line integral of $f$ only depends on a single curve $C_{a+he_j}$, which in turn depends on the point $a$. This potentially allows for multiple functions $f$ in $U$ for different curves and different $a$. Thus, the proof does not conclude that there exists a single $f$ satisfying $F(a) = \nabla f(a)$ everywhere in $U$.
\end{answer}


\end{parts}


 
\pagebreak
%%% QUESTION 4
\question \label{Irrotational} Irrotational vector fields are gradient vector fields in some cases. 
\begin{quote}
	\textbf{Theorem B.} \it If $U \subseteq \R^2$ is an open simply-connected set and $F$ is a $C^1$ irrotational vector field on  $U$, then $F$ is a gradient vector field on $U$. That is, $F = \nabla f$ on $U$ for some $C^2$ scalar function $f$ on $U$. 
\end{quote} 
On the other hand, you can verify that the vector field $ \ds F(x,y) = \big( \tfrac{-y}{x^2+y^2} , \tfrac{x}{x^2+y^2} \big)$ is irrotational and yet  $F$ is not a gradient vector field.  You may assume these facts without proof. 
\begin{parts}

\item Explain why $F$ does not contradict Theorem B in at most 2 full sentences. 
%% Do not change the height of this box; your work must fit inside
\begin{answer}{70pt}
% Insert your answer here
$F$ is not defined at the point $(0,0)$, so its domain is $\R \setminus \{(0,0)\}$, which is not a simply-connected set. Thus, Theorem B does not apply to $F$.
\end{answer}
\item \label{Upper} Let $V = \{ (x,y) \in \R^2 : y > 0\}$. Find all potential functions of the vector field  $F|_V$. 
%% Do not change the height of this box; your work must fit inside
\begin{answer}{430pt}
% Insert your answer here
Suppose $F|_V$ has a potential $f: V \to \R^2$, so $F|_V(x,y) = \nabla f(x,y)$ for $(x,y) \in V$. This means $f$ must satisfy $\tfrac{\partial f}{\partial x} = \tfrac{-y}{x^2+y^2}$ and $\tfrac{\partial f}{\partial y} = \tfrac{x}{x^2+y^2}$ for $(x,y) \in V$. (*)\\

Integrate the first equation with respect to $x$, holding $y$ fixed: $f(x,y) = \int \tfrac{-y}{x^2+y^2} dx$. Using the u-substitution $u = \tfrac{x}{y}$ and $du = \frac{dx}{y}$, this equals: $-\int \frac{y^2}{y^2u^2+y^2} du = -\int \frac{1}{u^2+1} du = -\arctan(u) + \phi(y) = -\arctan(\frac{x}{y}) + \phi(y)$, where $\phi: \R^+ \to \R^2$ is an arbitrary function of $y$.\\

Taking $f(x,y) = -\arctan(\frac{x}{y}) + \phi(y)$ as solved above, $\tfrac{\partial f}{\partial y} = (-\frac{y^2}{y^2+x^2})(-\frac{x}{y^2}) + \phi'(y) = \frac{x}{y^2+x^2} + \phi'(y)$. By (*), this means $\phi'(y) = 0$, so $\phi(y) = C$ for some $C \in \R$ by the MVT.\\

Since $\tfrac{\partial f}{\partial x} = \tfrac{-y}{x^2+y^2}$ and $\tfrac{\partial f}{\partial y} = \tfrac{x}{x^2+y^2}$ by direct calculation, all potential functions of $F|_V$ are represented by $f(x,y) = -\arctan(\frac{x}{y}) + C$ for $(x,y) \in V$, $C \in \R$.
\end{answer}
\pagebreak
\item \label{Lower} Let $W = \{(x,y) \in \R^2 : y < 0\}$. State (without proof) all potential functions of the vector field $F|_W$. 
\begin{answer}{60pt}
% Insert your answer here
$f(x,y) = -\arctan(\frac{x}{y}) + C$ for $(x,y) \in W$, $C \in \R$.
\end{answer}
\item Let $U = V \cup W \cup \{ (x,y) \in \R^2 : x > 0, y = 0\}$. Find a potential function $\phi : U \to \R$ of the vector field $F|_U$. Use (\ref{Irrotational}\ref{Upper}) and (\ref{Irrotational}\ref{Lower}) and additional arguments to justify that $F|_U = \nabla \phi$.  Can you extend your function $\phi$ to be continuous on a larger set containing $U$? Briefly explain why or why not. ({\color{Yellow} Revised 2023-03-13})
%% Do not change the height of this box; your work must fit inside
\begin{answer}{500pt}
% Insert your answer here
Define $\phi: U \to \R$ by \begin{equation} \phi(x,y) = \begin{cases} 
      f(x,y)+\frac{\pi}{2} & x \in \R, y>0\\
      f(x,y)-\frac{\pi}{2} & x \in \R, y<0\\
      \arctan(\frac{y}{x}) & x>0, y=0 
\end{cases}\end{equation} for $f$ as defined in 4b and 4c. Notice $\phi$ is continuous on $U$ since $f$ is continuous on $V$ and $W$, and for $x>0$, $\ds \lim_{y \to 0^-} f(x,0)-\frac{\pi}{2} = 0 = \ds \lim_{y \to 0^+} f(x,0)+\frac{\pi}{2} = \phi(x,0)$.\\

Since\begin{itemize}
  \item $\nabla \phi|_V = \nabla (f(x,y)+\frac{\pi}{2}) = \nabla f(x,y) = F|_V$ by 4b,
  \item $\nabla \phi|_W = \nabla (f(x,y)-\frac{\pi}{2}) = \nabla f(x,y) = F|_W$ by 4c, and
  \item $\nabla (\arctan(\frac{y}{x})) = (\tfrac{-y}{x^2+y^2}, \tfrac{x}{x^2+y^2}) = F(x,y)$ for $x>0,y=0$,
\end{itemize}
$\nabla \phi = F|_U$.\\

$\phi$ cannot be extended to be continuous on a larger set containing $U$ since for $x<0,y=0$, $\ds \lim_{y \to 0^-} f(x,0)-\frac{\pi}{2} = -\pi \neq \pi = \ds \lim_{y \to 0^+} f(x,0)+\frac{\pi}{2}$. Hence, it is impossible to join the pieces of $\phi$ with a function at the negative x-axis.
\end{answer}

\end{parts}
 


\pagebreak
%%% QUESTION 5
\question  Let $F = (f,g)$ be a vector field in $\R^2$ with $C^1$ components $f$ and $g$. Fix a point $p = (x,y) \in \R^2$ in the domain of $F$. For $\epsilon > 0$, let $B_{\epsilon}(p) \subseteq \R^2$ be the disk of radius $\epsilon$ centred at $p$. Orient its boundary $\partial B_{\epsilon}(p)$ counterclockwise.  \textbf{Do not use Green's theorem for any part of this question.} 

\begin{parts}	

\item \label{Setup} For $\epsilon > 0$, show that the flux of $F$ across $\partial B_{\epsilon}(p)$ may be expressed as
\[
\oint_{\partial B_{\epsilon}(p)} (F \cdot n) \,  ds = \int_{0}^{2\pi} f(x+\epsilon \cos t, y + \epsilon \sin t) \cdot \epsilon \cos t + g(x+ \epsilon \cos t, y + \epsilon \sin t) \cdot \epsilon \sin t dt.
\]

%% Do not change the height of this box; your work must fit inside
\begin{answer}{520pt}
% Insert your answer here
Let $\epsilon$ > 0. Parametrize $\partial B_{\epsilon}(p)$ by $\gamma: [0,2\pi] \to \R^2$ defined by $\gamma(t) = (x+\epsilon\cos t, y+\epsilon\sin t)$ for $t \in [0,2\pi]$. Notice $||\gamma'(t)|| = ||(-\epsilon\sin t, \epsilon\cos t)|| = \sqrt{(-\epsilon\sin t)^2 + (\epsilon\cos t)^2} = \sqrt{\epsilon^2} = \epsilon$.\\

The unit tangent of $\partial B_{\epsilon}(p)$ is $T(t) = \frac{\gamma'(t)}{||\gamma'(t)||} = \frac{(-\epsilon\sin t, \epsilon\cos t)}{\epsilon} = (-\sin t, \cos t)$.\\

Define $n: [0,2\pi] \to \R^2$ by $n(t) = (\cos t, \sin t)$ for $t \in [0,2\pi]$. Notice that for all $t \in [0,2\pi]$:\begin{itemize}
  \item $T(t) \cdot n(t) = (-\sin t)(\cos t) + (\cos t)(\sin t) = 0$.
  \item $||n(t)|| = \sqrt{\cos^2 t + \sin^2 t} = 1$.
  \item The matrix $(n(t), T(t)) = \begin{pmatrix}
  \cos t & -\sin t \\
  \sin t & \cos t
  \end{pmatrix}$ has determinant $1 > 0$, so $\{n(t), T(t)\}$ is a positively oriented basis in $\R^2$.
\end{itemize}
Thus, $n$ is the unit normal of $\partial B_{\epsilon}(p)$.\\

By definition, the flux of $F$ across $\partial B_{\epsilon}(p)$ is $\oint_{\partial B_{\epsilon}(p)} (F \cdot n) \,ds = \int_{0}^{2\pi} F(\gamma(t)) \cdot n(t)||\gamma'(t)|| dt = \int_{0}^{2\pi} F(x+\epsilon\cos t, y+\epsilon\sin t) \cdot (\cos t, \sin t)(\epsilon) dt = \int_{0}^{2\pi} f(x+\epsilon \cos t, y + \epsilon \sin t) \cdot \epsilon \cos t + g(x+ \epsilon \cos t, y + \epsilon \sin t) \cdot \epsilon \sin t dt$ as needed.
\end{answer}

\pagebreak
\item Since $f$ is $C^1$ on $U$,  differentiability implies that there exists $\delta_f> 0$ and  $E_f  : B_{\delta_f}((0,0)) \to \R$   such that
	\[
	\forall (\Delta x, \Delta y) \in B_{\delta_f}(0,0), \quad f(x+\Delta x, y+ \Delta y) = f(x,y) + \partial_1 f(x,y) \Delta x + \partial_2 f(x,y) \Delta y + E_f(\Delta x , \Delta y), 
	\] 
	where $\ds\lim_{(a,b) \to (0,0)} \tfrac{|E_f(a,b)|}{||(a,b)||}  = 0$. The analogous statement holds for $g$ with $\delta_g > 0$ and $E_g : B_{\delta_g}((0,0)) \to \R$. \\
	
	Prove that for $0 < \epsilon < \tfrac{\min\{ \delta_f, \delta_g \}}{2}$,   
	\[
  \frac{1}{\mathrm{area}(B_{\epsilon}(p))} \oint_{\partial B_{\epsilon}(p)} (F \cdot n)\, ds = (\Div F)(p) + \frac{1}{\pi \epsilon} \int_{0}^{2\pi} E_f(\epsilon \cos t, \epsilon \sin t) \cdot  \cos t + E_g(\epsilon \cos t, \epsilon \sin t) \cdot \sin t \; dt.  
	\]


%% Do not change the height of this box; your work must fit inside
\begin{answer}{470pt}
% Insert your answer here
\end{answer}

\pagebreak
\item Use the definition of a limit to prove that $\ds \lim_{\epsilon \to 0^+} \frac{1}{\pi\epsilon} \int_{0}^{2\pi} E_f(\epsilon \cos t, \epsilon \sin t) \cos t \; dt = 0$. Assuming without proof that a similar identity holds for $E_g$, conclude that  
\[
(\Div F)(p) = \ds \lim_{\epsilon \to 0^+} \frac{1}{\mathrm{area}(B_{\epsilon}(p))} \oint_{\partial B_{\epsilon}(p)} (F \cdot n)\, ds. 	
\]


%% Do not change the height of this box; your work must fit inside
\begin{answer}{550pt}
% Insert your answer here
\end{answer}

\end{parts}
 
\end{questions}

\end{document}
