\documentclass{exam}


%%% Declare commands whether new or old
\newcommand{\declarecommand}[1]{\providecommand{#1}{}\renewcommand{#1}}

%------------------------------------------------------------------------
%	COURSE INFORMATION
%------------------------------------------------------------------------

%%% Global document information
\declarecommand{\university}{University of Toronto}
\declarecommand{\coursename}{Multivariable Calculus with Proofs}
\declarecommand{\coursecode}{MAT237}
\declarecommand{\term}{2022-23}
\declarecommand{\firstterm}{Fall 2022}
\declarecommand{\secondterm}{Winter 2023}

\declarecommand{\title}{\coursecode\, \coursename}
\declarecommand{\subtitle}{\docname}
\declarecommand{\subsubtitle}{Due \dueday\, \duedate \, by \duetime}
\declarecommand{\author}{S. Artamonov, F. Parsch, and A. Zaman}


%%% URLs
\declarecommand{\urlGrading}{https://www.gradescope.ca/courses/7779} % Gradescope
\declarecommand{\urlLMS}{https://q.utoronto.ca/courses/280409} % Quercus 
\declarecommand{\urlFAQ}{https://q.utoronto.ca/courses/280409/pages/policies-and-faq} % Quercus FAQ page


%%%%%%%%%%%%%%%%%%%%%%%%%%%%%%%%%%%%%%%%%%%%%%%%%%%%%%%%%%%%%%%%%%%%%%%%%%%%%%%%%%%%%%%%%%%%%%%%%%%%
%%%		PACKAGES
%%%%%%%%%%%%%%%%%%%%%%%%%%%%%%%%%%%%%%%%%%%%%%%%%%%%%%%%%%%%%%%%%%%%%%%%%%%%%%%%%%%%%%%%%%%%%%%%%%%%

%------------------------------------------------------------------------
%	FORMAT AND LAYOUT
%------------------------------------------------------------------------
\usepackage[24hr,iso]{datetime} % Required to write time of compilation
\usepackage[margin=1in]{geometry} % Required for setting margins
\parindent0pt % Set indentation to 0 pt
\usepackage[english]{babel} % English language/hyphenation

\declarecommand{\toggledraft}{
	\usepackage[firstpageonly=true]{draftwatermark}
	\def\answerborderwidth{1pt}
}

%------------------------------------------------------------------------
%	COLORS
%------------------------------------------------------------------------
\usepackage{xcolor} % required to defined colours
	\definecolor{DarkBlue}{HTML}{25355A}
	\definecolor{LightBlue}{HTML}{007FA3}
	\definecolor{Yellow}{HTML}{A37500}
	\definecolor{Red}{HTML}{A3002E}

%------------------------------------------------------------------------
%	FONTS
%------------------------------------------------------------------------
\usepackage[charter,cal=cmcal]{mathdesign} % Specify font for serif family
\usepackage{avant} % Specify avant font for sans serif family
\usepackage{mathtools} % Replaces amsmath
\usepackage{sectsty} % Specify section formatting
	\sectionfont{\sffamily\color{DarkBlue}}
	\subsectionfont{\sffamily\color{LightBlue}}
	\subsubsectionfont{\small\sffamily}

%------------------------------------------------------------------------
%	TABLES, IMAGES, AND FIGURES
%------------------------------------------------------------------------

\usepackage{array} % Required for tables
\usepackage{graphicx} % Required for including images
	\graphicspath{{img/}} % Specifies the directory where pictures are stored
\usepackage{tikz} % Required for creating plots
	\usetikzlibrary{decorations.pathreplacing}
\usepackage{tikz-3dplot} % Required for 3D plots
\usepackage{tikz-cd} % Required for commutative diagrams
\usepackage{pgfplots}
	\pgfplotsset{compat=1.12}
	\usepgfplotslibrary{colormaps}
	\usepgfplotslibrary{patchplots}
	\usepgfplotslibrary{fillbetween}

%------------------------------------------------------------------------
%	LINKS AND REFERENCES
%------------------------------------------------------------------------

\usepackage[hidelinks, urlcolor=Red, linkcolor=Yellow, colorlinks=true]{hyperref}   
\usepackage[noabbrev,capitalise,nameinlink]{cleveref}

%------------------------------------------------------------------------
%	BOXES, CHOICES, AND LISTS
%------------------------------------------------------------------------

\usepackage{enumerate}
\usepackage{enumitem} % Customize lists
% \setlist{nolistsep} % Reduce spacing between bullet points and numbered lists

\declarecommand\partlabel{(\thequestion\alph{partno})} % parts are labelled (1a)

\usepackage{environ}
%%% Answer environment makes a framed box of a chosen height
\NewEnviron{answer}[1]{
	\fullwidth{
	\vspace*{-10pt}
		\par\nobreak\vspace{\ht\strutbox}\noindent
		\setlength{\fboxrule}{\answerborderwidth}
		\fbox{% 
		\parbox[c][#1][t]{\dimexpr\linewidth-2\fboxsep}{
		\hrule width \hsize height 0pt \color{LightBlue} 
		\BODY
		}%
		}%
	}
}
\def\answerborderwidth{0pt} % Set answer boxes to be invisible by default

% Define remark environment
\NewEnviron{remark}{
	\fullwidth{
	\vspace*{-10pt}
		\color{Red} \textbf{Remark:}  
		\BODY
	}
}

\CorrectChoiceEmphasis{\color{LightBlue}\bf}	

\declarecommand{\CorrectChoiceBox}{
	\checkboxchar{$\Box$} 
	\checkedchar{$\blacksquare$}
}
\declarecommand{\CorrectChoiceCircle}{
	\checkboxchar{\tikz[baseline=-0.6ex]\draw[black,thick] (0,0) circle (1ex);}
	\checkedchar{\tikz[baseline=-0.6ex]\draw[LightBlue,fill=LightBlue,thick] (0,0) circle (1ex);}
}

 
%------------------------------------------------------------------------
%	THEOREMS
%------------------------------------------------------------------------

\usepackage{amsthm}

%%% Specify theorems
\newtheorem{theorem}{Theorem}
\newtheorem*{definition}{Definition}

\newtheoremstyle{named}{}{}{\itshape}{}{\bfseries}{.}{.5em}{\thmnote{#3}}
\theoremstyle{named}
\newtheorem*{namedtheorem}{Theorem}


%% Setup proof lines
\NewEnviron{lines}{
	\vspace*{-5pt}
	\begin{center}
	\fbox{\begin{minipage}{0.9\textwidth}
	\begin{enumerate}[label=\arabic*.]
		\it \BODY
	\end{enumerate}
	\end{minipage}
	}
	\end{center}
}
 
%------------------------------------------------------------------------
%	HEADER & FOOTER
%------------------------------------------------------------------------

%%% Define footnote without marker
\declarecommand{\blfootnote}[1]{%
  \begingroup
  \declarecommand\thefootnote{}\footnote{#1}%
  \addtocounter{footnote}{-1}%
  \endgroup
}

%%% Specify header and footer
\pagestyle{headandfoot} 	\runningfootrule  
	\runningfooter{\coursecode}{\docname\ - Page \the\numexpr\thepage\ of \numpages}{\duedate}

%------------------------------------------------------------------------
%	TITLE PAGE
%------------------------------------------------------------------------


%%% Define title
\declarecommand{\maketitle}{
	\begin{center}
		{\color{DarkBlue}\Large\sffamily\bfseries\title} \\[2pt]
		{\color{LightBlue}\Large\sffamily\bfseries\subtitle}
		\blfootnote{Created and revised by \author. Last updated \today \, at \currenttime.}\\
		{\sffamily\subsubtitle} 
	\end{center}
 
}

%%% Define title page
\declarecommand{\makeinstructions}
{ 
\section*{Instructions}
This problem set is based on \docinfo. Please read the \href{\urlFAQ}{Problem Set FAQ} for details on submission policies, collaboration  rules, and general instructions.  
\begin{itemize}
	\item \textbf{\href{\urlLMS/pages/tutorials}{Tutorials} on \tutorialdate\, will help you with \docname.} You will work with peers and get help from TAs. Before attending, seriously attempt these problems and prepare initial drafts.
	\item \textbf{Submissions are only accepted by \href{\urlGrading}{Gradescope}}. Do not send anything by email.  Late submissions are not accepted under any circumstance. Remember you can resubmit anytime before the deadline. 
	\item \textbf{Submit your polished solutions using only this template PDF.} You will submit a single PDF with your full written solutions. If your solution is not written using this template PDF (scanned print or digital) then you will receive zero. Do not submit rough work. Organize your work neatly in the space provided.  
	\item \textbf{Show your work and justify your steps} on every question, unless otherwise indicated. Put your final answer in the box provided, if necessary. 
\end{itemize}
We recommend you write draft solutions on separate pages and afterwards write your polished solutions here. You must fill out and sign the academic integrity statement below; otherwise, you will receive zero. 

\section*{Academic integrity statement}
%%% Student information
% Person A
\fbox{
\begin{minipage}{\textwidth}
{
	\vspace{0.2in}
	
	\makebox[\textwidth]{\sffamily Full Name:\enspace{\bfseries\large\FirstStudentName\,}\hrulefill}
	
	\vspace{0.2in}
	
	\makebox[\textwidth]{\sffamily Student number:\enspace{\bfseries\large\FirstStudentNumber\,}\hrulefill}
	
	\vspace{0.1in}
	
}
\end{minipage}
}

\vspace*{0.1in}

% Person B
\fbox{
\begin{minipage}{\textwidth}
{
\vspace{0.2in}

\makebox[\textwidth]{\sffamily Full Name:\enspace{\bfseries\large\SecondStudentName\,}\hrulefill}

\vspace{0.2in}

\makebox[\textwidth]{\sffamily Student number:\enspace{\bfseries\large\SecondStudentNumber\,}\hrulefill}

\vspace{0.1in}

}
\end{minipage}
}
~

I confirm that:

\begin{itemize} 
	\item I have read and followed the policies described in the \href{\urlFAQ}{Problem Set FAQ}. 
	\item I have read and understand the rules for collaboration on problem sets described in the Academic Integrity subsection of the syllabus. I have not violated these rules while writing this problem set. 
	\item I understand the consequences of violating the University's academic integrity policies as outlined in the \href{http://www.governingcouncil.utoronto.ca/policies/behaveac.htm}{Code of Behaviour on Academic Matters}. I have not violated them while writing this assessment.
\end{itemize}
By signing this document, I agree that the statements above are true. 

% You should sign this PDF after compiling. Do not write your signature using LaTeX.
\vspace{0.2in}
{\large 
	\makebox[\textwidth]{\sffamily Signatures: 1)\enspace\hrulefill} 
	
	\vspace{0.2in}
	
	\makebox[\textwidth]{\sffamily \hspace*{22mm} 2)\enspace\hrulefill} 
}
\vfill
\pagebreak
}



%%% Uncomment if you are writing a draft
%\toggledraft

%%% Uncomment if you want to show your multiple choice answers
\printanswers
\usepackage{graphicx}
%%% Local document information
\declarecommand{\docname}{Problem Set 8}		\usepackage{neuralnetwork}
\declarecommand{\dueday}{Friday}
\declarecommand{\duedate}{March 31, 2023}
\declarecommand{\duetime}{13:00 ET}
\declarecommand{\tutorialdate}{Tuesday March 28, 2023}
\declarecommand{\docinfo}{\href{\urlFAQ/modules}{Module K: Fundamental theorems in 2D} (K2 and K3) and \href{\urlFAQ/modules}{Module L: Surface integrals} (L1 to L4)}

%%% Uncomment both if you are a MAT237 teaching member writing the official  solutions
%\declarecommand{\makeinstructions}{\vfill\pagebreak}
%\declarecommand{\docname}{Problem Set 7 solutions}
%\declarecommand{\authors}{C. Davies, A. Qiu, and A. Zaman}

%%% Student information
\declarecommand{\FirstStudentName}{} 
\declarecommand{\FirstStudentNumber}{}

\declarecommand{\SecondStudentName}{} % leave empty if you are submitting individually
\declarecommand{\SecondStudentNumber}{} % leave empty if you are submitting individually

%%% Custom math commands
%% Standard
\declarecommand{\ds}{\displaystyle} % force math styling
\declarecommand{\emptyset}{\varnothing} % better empty set
\declarecommand{\epsilon}{\varepsilon} % better epsilon
\declarecommand{\R}{\mathbb{R}} % reals
\declarecommand{\Z}{\mathbb{Z}} % integers
\declarecommand{\Q}{\mathbb{Q}} % rationals
\declarecommand{\N}{\mathbb{N}} % naturals
\declarecommand{\C}{\mathbb{C}} % complex
\DeclarePairedDelimiter\abs{\lvert}{\rvert} % absolue value
\DeclarePairedDelimiter\norm{\lVert}{\rVert} % norm

%% Linear algebra
\declarecommand{\mat}[1]{\begin{bmatrix*}[r]#1\end{bmatrix*}}
\declarecommand{\matc}[1]{\begin{bmatrix}#1\end{bmatrix}} % matrix
\DeclareMathOperator{\Span}{span} % span
\DeclareMathOperator{\Img}{img} % image
\DeclareMathOperator{\Ker}{ker} % kernel
\DeclareMathOperator{\Id}{id} % identity map
\DeclareMathOperator{\Range}{range} % range
\DeclareMathOperator{\Rref}{rref} % row reduced echelon form
\DeclareMathOperator{\Rank}{rank} % rank
\DeclareMathOperator{\Null}{null} % null space
\DeclareMathOperator{\Nullity}{nullity} % nullity
\DeclareMathOperator{\Proj}{proj} % projection
\DeclareMathOperator{\Dim}{dim} % dimension

%% Topology
\DeclareMathOperator{\Int}{int}
\DeclareMathOperator{\Cl}{cl}

%% Calculus
\DeclareMathOperator{\Length}{length} % length
\DeclareMathOperator{\Area}{area} % area
\DeclareMathOperator{\Vol}{vol} % volume
\DeclareMathOperator{\Grad}{grad} % gradient
\DeclareMathOperator{\Curl}{curl} % curl
\DeclareMathOperator{\Div}{div} % divergence

\begin{document}



\maketitle

\vspace*{-15pt}
\makeinstructions
 
%%% PROBLEMS
\section*{Problems} 

\CorrectChoiceCircle % for multiple choice questions

%%% Do not change the height of this box; your work must fit inside
%\begin{answer}{550pt}
%% Insert your answer here
%\end{answer}


\begin{questions}

%%% QUESTION 1
\question  Let $F(x,y) = \left( xy^2+2y, e^{x^2} + y^4   \right)$ be a vector field in $\R^2$. Let $C_1 \subseteq \R^2 $ be the oriented curve  parametrized by $\gamma_1(t) = (6-t-t^2, t)$ for $-3 \leq t \leq 2$. Compute the normal flow of $F$ across $C_1$ and include a well-labeled sketch illustrating your argument. \textit{Hint:} Close the loop. 

% Do not change the height of this box
\begin{answer}{560pt}
% Insert your answer here
Define $C_2$ as the straight line from $(0,-3)$ to $(0,2)$. Parametrize it by $\phi: [-3,2] \to \R^2$ defined by $\phi(t) = (0,t)$ for $t\in [-3,2]$. The tangent to $C_2$ is $T_{C_2}(t) = (0,1)$ for $t \in [-3,2]$. Define $n:[-3,2]\to\R^2$ by $n(t)=(1,0)$ for $t\in[-3,2]$. Notice $n$ is the normal to $C_2$ since for $t \in [-3,2]$, $n(t) \cdot T_{C_2} = 0$, $||n(t)|| = 1$, and $\{n(t), T_{C_2}\}$ is a positively oriented basis of $\R^2$ by the right hand rule.\\

Define the region $R = \{(x,y)\in\R^2: -3\leq y\leq 2,0\leq x\leq 6-y-y^2\}$. A picture of $C_1$, $C_2$, and $R$ with their orientations is shown below:

%%% Link to graph: https://www.desmos.com/calculator/roqkv7zhh4
\begin{center}
    \includegraphics[scale=0.15]{mat237-ps8-tex/Graph.png}
\end{center}

Notice $C_1-C_2 = \{(x,y)\in\R^2:x=6-y-y^2, y\in[-3,2]\}\cup\{(0,y)\in\R^2:y\in[-3,2]\} = \partial R$ is a positively oriented piecewise curve. Also, $F$ is $C_1$ on $R$, which is a regular region since $R = \overline{R^{\mathrm{o}}}$. Thus, Green's theorem (divergence form) can be applied:
\begin{align}
\oint_{\partial R} (F\cdot n) ds = \oint_{C_1-C_2} (F\cdot n) ds = \int_{C_1} (F\cdot n) ds - \int_{C_2} (F\cdot n) ds = \iint_R \Div(F) dA
\end{align} by Lemma 11.3.16.\\

Then, we have the following:\begin{align*}\int_{C_2} (F\cdot n)ds = \int_{-3}^{2} F(\phi(t))\cdot n(t)||\phi'(t)||dt = \int_{-3}^{2} (2t,1+t^4)\cdot (1,0)dt = \int_{-3}^{2}2tdt = -5\end{align*}\begin{align*}\iint_R \Div(F)dA = \iint_R (y^2+4y^3)dA = \int_{-3}^2\int_0^{6-y-y^2} (y^2+4y^3)dxdy=\int_{-3}^2 (6y^2+23y^3-5y^4-4y^5)dy =-\frac{1625}{12}\end{align*}\\

Thus, by rearranging (1), we have \begin{align*}\int_{C_1}(F\cdot n)ds = -\frac{1625}{12}+(-5) = -\frac{1685}{12}\end{align*}
\end{answer}

\pagebreak
%%% QUESTION 2
\question   \label{AllTheCalculus} Multivariable calculus has shown how you can do calculus with all of your linear algebra. Now, near the end of your journey, it is time to do linear algebra with all of your  calculus (in two dimensions). 

Let $U \subseteq \R^2$ be an open set.  Let $C^{\infty}(U)$ be the set of real-valued functions $f : U \to \R$  with infinitely many partial derivatives; that is, $\partial^{\alpha} f$ exists and is continuous on $U$ for all multi-indices $\alpha \in \N^2$. The space of \textbf{$C^{\infty}$ scalar functions} $V= C^{\infty}(U)$ and space of \textbf{$C^{\infty}$ vector fields} $V^2 = V \times V$ can each be thought of as a space of vectors.  For example, the zero function belongs to $V$  and acts  as the zero vector. Moreover, any linear combination in $V$ also belongs to $V$. Similar statements hold  true for $V^2$.   


\begin{parts}
	\item   You can view the differential operators `$\Grad$' and `$\Curl$' as linear transformations on these spaces. 
	\begin{itemize}
		\item Gradient is a linear map of $C^{\infty}$ scalar functions to $C^{\infty}$ vector fields. 
	
		That is, $\mathrm{grad} : V \to V^2$ is a linear map. Hence, if $f \in V$ then $\Grad(f) \in V^2$.	
	
		\item Two-dimensional curl is a linear map of $C^{\infty}$ vector fields to $C^{\infty}$ scalar functions. 
	
		That is, $\Curl : V^2 \to V$ is a linear map. Hence, if $F \in V^2$ then $\Curl(F) \in V$. 
	\end{itemize}
	Prove that  $\Curl$ is a linear map from $V^2$ to $V$. In other words, show $\Curl(F + \lambda G) = \Curl(F) + \lambda \Curl(G)$ for any $F, G \in V^2$ and any $\lambda \in \R$.  You may assume that the  partial derivative operators $\partial_1 : V \to V$ and $\partial_2 : V \to V$ are linear maps. 
	% Do not change the height of this box
	\begin{answer}{250pt}
		% Insert your answer here
Let $F, G \in V^2$ and $\lambda \in \R$. $\Curl(F + \lambda G) = \partial_1 ((F + \lambda G)_2) - \partial_2 ((F + \lambda G)_1) = \partial_1 (F_2 + \lambda G_2) - \partial_2 (F_1 + \lambda G_1) = \partial_1 F_2 + \lambda \partial_1 G_2 - \partial_2 F_1 - \lambda \partial_2 G_1$ since $\partial_1$ and $\partial_2$ are linear maps. Rearranging terms, this equals $(\partial_1 F_2 - \partial_2 F_1) + \lambda(\partial_1 G_2 - \partial_2 G_1) = \Curl(F) + \lambda\Curl(G)$.
	\end{answer}	
	
	\item \label{ImgKer} The image of the gradient is contained in the kernel of curl. That is, $\Img(\Grad) \subseteq \Ker (\Curl)$. There are two ways to prove this fact: by boring calculation or by the "one true proof". 
	
	Prove that $\Img(\Grad) \subseteq \Ker (\Curl)$ by a boring calculation with partial derivatives.  
	% Do not change the height of this box
	\begin{answer}{120pt}
		% Insert your answer here
Let $f\in V$, so $\Grad(f)=(\partial_1 f,\partial_2 f) \in V^2$. By Clairaut's theorem (Theorem 6.2.8), since $U$ is open and $f$ is $C^1$, $\partial_2\partial_1 f=\partial_2\partial_1 f$. Thus, $\Curl(\Grad(f)) = \partial_1\partial_2 f-\partial_2\partial_1 f=0$, meaning $\Grad(f)\in\Ker(\Curl)=\{F\in V^2:\Curl(F)=0\}$. Since $f$ was arbitrary, $\Img(\Grad)\subseteq\Ker(\Curl)$.
	\end{answer}	


	\pagebreak
	\item The "one true proof" of $\Img(\Grad) \subseteq \Ker (\Curl)$  relies upon the fundamental theorem of line integrals and Green's theorem.  Here is such an attempt to prove this containment. 
	\begin{lines}
		\item Let $F \in \Img(\Grad)$ and $p \in U$. Then $\forall \epsilon > 0$,$\ds\oint_{\partial B_{\epsilon}(p)} (F \cdot T)\,  ds = 0$.
		\item  $\ds \implies \forall \epsilon > 0$, $\ds \iint_{\overline{B_{\epsilon}(p)}} (\Curl F) \, dA = 0  \implies \lim_{\epsilon \to 0^+} \left[ \frac{1}{\Area(\overline{B_{\epsilon}(p)})} \iint_{\overline{B_{\epsilon}(p)}} (\Curl F) \, dA \right] = 0$
		\item $\ds  \implies  (\Curl F)(p) = 0 \implies F \in \Ker(\Curl)$ 
	\end{lines}
	There are no serious errors but it is terribly written. Rewrite this into a well-written justified proof. \\ Do \textbf{not} use Lemma 12.1.7.   \textit{Hint:} Use FTLI, Green's, and integral MVT. 

	% Do not change the height of this box
	\begin{answer}{470pt}
		% Insert your answer here
Let $F \in \Img(\Grad)$, so $F=\Grad(f)$ for some $f\in V$. Let $\epsilon>0$, and let $p=(x,y)\in U$. Define $R=\overline{B_{\epsilon}(p)}$ and $\partial R=\partial B_{\epsilon}(p)$.\\

Parametrize $\partial R$ by $\gamma:[0,2\pi] \to \R^2$ defined by $\gamma(t)=(x+\epsilon\cos(t),y+\epsilon\sin(t))$ for $t\in[0,2\pi]$. Notice that $\partial R=\gamma([0,2\pi])$, $\gamma$ is continuous on $[0,2\pi]$, $\gamma$ is $C^1$ on $(0,2\pi)$, $\gamma '(t)=(-\epsilon\sin(t),\epsilon\cos(t))\neq0$ for $t\in(0,2\pi)$, and $\gamma$ is injective on $(0,2\pi)$. By definition 11.1.8, $\gamma$ is a simple regular parametrization of $\partial R$, meaning $\partial R$ is a curve. Furthermore, $\gamma(0)=\gamma(2\pi)$, so $\partial R$ is closed.\\

By definition, the circulation of $F$ along $\partial R$ is $\oint_{\partial R} F\cdot Tds=\int_0^{2\pi} F(\gamma(t))\cdot\gamma'(t)dt=\int_{\partial R} F\cdot d\gamma$, where $d\gamma=\gamma'(t)dt$. Since $F=\Grad(f)$, this is equivalent to $\int_{\partial R} \Grad(f)\cdot d\gamma$, which by the FTLI equals $f(\gamma(0))-f(\gamma(2\pi))=f(x+\epsilon,y)-f(x+\epsilon,y)=0$. (The FTLI can be applied since $f$ is $C^1$ on U, which contains $\partial R$.) Thus, $\oint_{\partial R} F\cdot Tds=0$.\\

Note $R$ is compact, Jordan measurable, and $\overline{R^{\mathrm{o}}}=R$, so it is a regular region. By Problem Set 7 5a), the unit normal of $\partial R$ is $n(t)=(\cos(t),\sin(t))$ for $t\in[0,2\pi]$, which points away from $R$, so $\partial R$ is positively oriented. Thus, Green's theorem (curl form) can be applied: $\oint_{\partial R} F\cdot Tds=\iint_{R} \Curl(F) dA=0$. (*)\\

Note (*) holds true for all $\epsilon>0$. Thus, \begin{align}
    \lim_{\epsilon \to 0^+} \left[ \frac{1}{\Area(R)} \iint_R (\Curl F) \, dA \right] = \lim_{\epsilon \to 0^+} \frac{0}{\Area(R)}=0
\end{align}since $\Area(R)=\Area(\overline{B_{\epsilon}(p)})>0$. By Corollary 8.2.8 (as derived from the integral MVT), since $R=\overline{B_{\epsilon}(p)}$ and $\Curl F$ is continuous, (2) is equal to $(\Curl F)(p)$. Hence, $(\Curl F)(p)=0$, and since $p$ was arbitrary in $U$, $F\in\Ker(\Curl)=\{F\in V^2:\Curl(F)=0\}$. Since $F$ was arbitrary in $\Img(\Grad))$, $\Img(\Grad)\subseteq\Ker(\Curl))$.
\end{answer}	
	
	
	\pagebreak
	\item The image may or may not equal the kernel in (\ref{AllTheCalculus}\ref{ImgKer}). It depends on the topology of $U \subseteq \R^2$.  
	
	\begin{itemize}
		\item Give an example of a set $U = U_1$ where $\Img(\Grad)  = \Ker(\Curl)$.
		\item Give an example of a set $U = U_2$ where $\Img(\Grad) \subsetneq \Ker(\Curl)$.
	\end{itemize}
	Briefly justify each of your examples. \textit{Hint:} Check out the J5 readings and/or worksheet.
	
	% Do not change the height of this box
	\begin{answer}{250pt}
		% Insert your answer here
Define $U_1=R^2$, which is a simply connected domain since it is open, path-connected, and the inside of every simple closed curve lying in $U_1$ is a subset of $U_1$. By Theorem 11.5.14, since $U_1$ is simply connected, every irrotational function $F\in V^2$ is conservative on $U_1$. Thus, $\forall F\in V^2$, $F\in \Ker(\Curl)\implies F$ is irrotational $\implies\exists f: U_1\to \R$ such that $F=\Grad(f)$ by Theorem 11.5.14. This implies $\Ker(\Curl)\subseteq\Img(\Grad)$ for $U_1$, so $\Img(\Grad) = \Ker(\Curl)$.\\

Define $U_2: R^2\setminus\{(0,0,0)\}$ and $F:U_2\to\R^2$ by $F(x,y)=(\frac{-y}{x^2+y^2},\frac{x}{x^2+y^2})$ for $(x,y)\in U_2$. Notice that $\partial_1 F_2=\frac{y^2-x^2}{(y^2+x^2)^2}=\partial_2 F_1$, so $F$ is irrotational and $F\in \Ker(\Curl)$. However, since $F$ is not a gradient vector field (hence not conservative) by Problem Set 7 Q4, there is no function $f\in V$ such that $F=\Grad(f)$. Thus, $\Ker(\Curl)\nsubseteq\Img(\Grad)$, so $\Img(\Grad) \subsetneq \Ker(\Curl)$.
	\end{answer}	
	
	\item These observations about $\Grad$ and $\Curl$   can be beautifully encapsulated in this elegant diagram. 
	\begin{center}
	\adjustbox{scale=1.5,center}{
	\begin{tikzcd}
 	 V \arrow{r}{\Grad}  & V^2 \arrow{r}{\Curl}  & V 
	\end{tikzcd}
	}
	\end{center}
	At first glance, this appears to just be a composition of maps but if you dig a bit deeper, you will notice it actually captures a lot of vector calculus in $\R^2$. 	Take an arbitrary element at the leftmost $V$ in the diagram. Map it once to $V^2$ and then map it again to the next $V$. The element has moved two stages to the right. What happened to this element? How do the two core theorems of vector calculus in $\R^2$ relate to this phenomenon? Explain in two to three full sentences using the previous parts of this question. 


	% Do not change the height of this box
	\begin{answer}{200pt}
		% Insert your answer here
For every element $f\in V$, its image under $\Grad$, $\Grad(f)\in V^2$, is conservative, and the image of $\Grad(f)$ under $\Curl$ is the zero function in $V$. This follows from FTLI and Green's by 2c, thus showing that conservative vector fields are irrotational (encapsulated in the $V^2\to V$ transformation) and creating this phenomenon.
	\end{answer}	
\end{parts}


\pagebreak
%%% QUESTION 3
\question  Fix $R, H > 0$. Let $S\subseteq \R^3$ be the part of the cone $H^2(x^2+y^2) = R^2 z^2$ with $0 \leq z \leq H$. 
\begin{parts}
	\item Prove that $S$ is a surface. 
	% Do not change the height of this box
	\begin{answer}{580pt}
		% Insert your answer here
Define $D = [0,R]\times[0,2\pi]$ and $\phi: D \to \R^2$ by $\phi(r,t)=(r\cos t, r\sin t, \frac{H}{R}r)$. Note $D$ is regular since it is compact, Jordan measurable, and $D = \overline{D^{\mathrm{o}}}$. Also, $D$ is path-connected, $\phi$ is continuous, and $\phi(D) = S$. Thus, by definition 13.1.1, $\phi$ is a (2-variable) parametrization of $S$.\\

For $p\in D^{\mathrm{o}}$, $\phi$ is $C^1$ at $p$ and $\{\partial_1\phi(p),\partial_2\phi(p)\}=\{(\cos t, \sin t, \frac{H}{R}), (-r\sin t, r\cos t, 0)\}$ is a linearly independent set. Thus, by definition 13.1.7, $\phi$ is regular.\\

To show $\phi$ is injective on $D$ except along $\partial D$, suppose that $\exists x,y\in D$ such that $x,y\notin\partial D$, $\phi(x)=\phi(y)$ and $x\neq y$. This yields the system of equations\begin{align*}
    x_1\cos(x_2) &= y_1\cos(y_2)\\
    x_1\sin(x_2) &= y_1\sin(y_2)\\
    \frac{H}{R}x_1 &= \frac{H}{R}y_1
\end{align*}By the third equation, $x_1=y_1$. Since $x\neq y$, this means $x_2\neq y_2$. By the first and second equations, $\cos(x_2)=\cos(y_2)$ and $\sin(x_2)=\sin(y_2)$, meaning either $x_2=0,y_2=2\pi$ or $x_2=2\pi,y_2=0$. This yields a contradiction since $(x,y)\in(\{(x_1,0)\in\R^2:x_1\in[0,2\pi]\}\cup\{(x_1,2\pi)\in\R^2:x_1\in[0,2\pi]\})\subseteq\partial D$, or else $x_2=y_2$ which implies $x=y$. Thus, $\forall x,y\in D$, either $\phi(x)=\phi(y)\implies x=y$ or $x,y\in\partial D$, meaning $\phi$ is simple by definition 13.1.12.\\

By definition 13.1.18, $S$ is a surface since $\phi$ is a simple regular 2-variable parametrizaton of $S$.
	\end{answer}
	
	\pagebreak
	\item Calculate the surface area of $S$ by definition. 
	% Do not change the height of this box
	\begin{answer}{600pt}
		% Insert your answer here
	\end{answer}

	\pagebreak
	\item Orient $S$ with upward unit normal. Find the flux of $F(x,y,z) = (x,y,z)$ across $S$ by directly calculating the surface integral $\iint_S F \cdot n \; dS$. Do not use Stokes' theorem  or the divergence theorem.
 
% Do not change the height of this box
\begin{answer}{600pt}
	% Insert your answer here
By definition 13.4.1 and using the cross product calculation from part b,\begin{equation*}\begin{split}
    \iint_S F \cdot n \; dS&=\iint_D (F\circ\phi)\cdot(\partial_1\phi\times\partial_2\phi)\;dA \\
    &=\int_0^{2\pi}\int_0^R F(\phi(r,t))\cdot(-\frac{H}{R}r\cos(t),-\frac{H}{R}r\sin(t),r)\; drdt \\
    &=\int_0^{2\pi}\int_0^R (r\cos(t),r\sin(t),\frac{H}{R}r)\cdot(-\frac{H}{R}r\cos(t),-\frac{H}{R}r\sin(t),r)\;drdt \\
    &=\int_0^{2\pi}\int_0^R -\frac{H}{R}r^2\cos^2(t)-\frac{H}{R}r^2\sin^2(t)+\frac{H}{R}r^2 \;drdt \\
    &=\frac{H}{R} \int_0^{2\pi}\int_0^R r^2-r^2 \;drdt \\
    &=\frac{H}{R} \int_0^{2\pi}\int_0^R 0 \;drdt \\
    &=0
\end{split}\end{equation*}Note that in the first to second lines of the above, $(\partial_1\phi\times\partial_2\phi)$ was chosen to be (+1)$(-\frac{H}{R}r\cos(t),-\frac{H}{R}r\sin(t),r)$. This is since the unit normal points upwards, which is consistent with the direction of $(\partial_1\phi\times\partial_2\phi)$ (which has a positive z component), meaning its sign is positive.
\end{answer}

\end{parts}





\pagebreak
%%% QUESTION 5
\question  Let $G : A \to \R^3$ and $H : B \to \R^3$ be parametrizations of an oriented surface $S \subseteq \R^3$. Assume there exists a diffeomorphism $\varphi : U \to V$ such that $\det D\varphi > 0$, $A \subseteq U$, $B \subseteq V$, $B = \varphi(A)$,  and $G(u) = (H \circ \varphi)(u)$ for $u \in A$.  
\begin{parts}

\item Let $F$ be a vector field in $\R^3$ that is continuous on $S$. Assuming 	
\begin{equation} \label{cross}
\forall u \in A^o, \quad (\partial_1 G \times \partial_2 G)(u)  =   (\partial_1 H \times \partial_2H)(\varphi(u)) \det D\varphi(u),
\tag{$\star$} 		
\end{equation}
prove that 
\[
\iint_A (F \circ G)  \cdot (\partial_1 G \times \partial_2 G) \; dA =	\iint_B (F \circ H)  \cdot (\partial_1 H \times \partial_2 H) \; dA 
\]
provided both integrals exist. In other words, the surface integral is invariant under reparametrization. 
% Do not change the height of this box
\begin{answer}{500pt}
	% Insert your answer here
Assume both integrals exist. Define $\gamma: B^{\mathrm{o}}\to\R$ by $\gamma(v)=(F\circ H)(v)\cdot(\partial_1 H \times \partial_2 H)(v)$ for $v\in B^{\mathrm{o}}$. Thus, for $u\in A^{\mathrm{o}},$\begin{align}
    \gamma(\varphi(u))=(F\circ H)(\varphi(u))\cdot(\partial_1 H \times \partial_2 H)(\varphi(u))
\end{align}

For $u\in A^{\mathrm{o}}$, $(F \circ G)(u)  \cdot (\partial_1 G \times \partial_2 G)(u) = (F \circ (H \circ \varphi))(u)\cdot(\partial_1 H \times \partial_2 H)(\varphi(u))\det D\varphi(u)$ by (*). Notice that by (3), this becomes $\gamma(\varphi(u))\det D(\varphi(u))=\gamma(\varphi(u))|\det D(\varphi(u))|$ since $\det D(\varphi(u))>0$.\\

Since $G$ is a simple regular parametrization, its domain $A$ is a regular region and hence a compact Jordan measurable set. Notice $\gamma$ is integrable on $\varphi(A)=B$ since $\iint_B (F \circ H)  \cdot (\partial_1 H \times \partial_2 H) \; dA$ exists, so by the change of variables theorem, $(\gamma\circ\varphi)|\det D\varphi|$ is integrable on $A$ and the following holds:\begin{equation*}\begin{split}
    \iint_A (F \circ G)  \cdot (\partial_1 G \times \partial_2 G) \; dA &=\iint_A (\gamma\circ\varphi)|\det D\varphi| \; dA \\
    &=\iint_{\varphi(A)}\gamma \; dA\\
    &=\iint_B (F\circ H)\cdot(\partial_1 H \times \partial_2 H) dA
\end{split}\end{equation*} by definition of $\gamma$ and $\varphi(A) = B$.
\end{answer}		

\item By applying and citing properties of the cross product, prove that \eqref{cross} holds. 

% Do not change the height of this box
\begin{answer}{600pt}
	% Insert your answer here
\end{answer}

\end{parts} 
 



\end{questions}

\end{document}
