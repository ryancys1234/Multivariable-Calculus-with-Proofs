 \documentclass{exam}


%%% Declare commands whether new or old
\newcommand{\declarecommand}[1]{\providecommand{#1}{}\renewcommand{#1}}

%------------------------------------------------------------------------
%	COURSE INFORMATION
%------------------------------------------------------------------------

%%% Global document information
\declarecommand{\university}{University of Toronto}
\declarecommand{\coursename}{Multivariable Calculus with Proofs}
\declarecommand{\coursecode}{MAT237}
\declarecommand{\term}{2022-23}
\declarecommand{\firstterm}{Fall 2022}
\declarecommand{\secondterm}{Winter 2023}

\declarecommand{\title}{\coursecode\, \coursename}
\declarecommand{\subtitle}{\docname}
\declarecommand{\subsubtitle}{Due \dueday\, \duedate \, by \duetime}
\declarecommand{\author}{S. Artamonov, F. Parsch, and A. Zaman}


%%% URLs
\declarecommand{\urlGrading}{https://www.gradescope.ca/courses/7779} % Gradescope
\declarecommand{\urlLMS}{https://q.utoronto.ca/courses/280409} % Quercus 
\declarecommand{\urlFAQ}{https://q.utoronto.ca/courses/280409/pages/policies-and-faq} % Quercus FAQ page


%%%%%%%%%%%%%%%%%%%%%%%%%%%%%%%%%%%%%%%%%%%%%%%%%%%%%%%%%%%%%%%%%%%%%%%%%%%%%%%%%%%%%%%%%%%%%%%%%%%%
%%%		PACKAGES
%%%%%%%%%%%%%%%%%%%%%%%%%%%%%%%%%%%%%%%%%%%%%%%%%%%%%%%%%%%%%%%%%%%%%%%%%%%%%%%%%%%%%%%%%%%%%%%%%%%%

%------------------------------------------------------------------------
%	FORMAT AND LAYOUT
%------------------------------------------------------------------------
\usepackage[24hr,iso]{datetime} % Required to write time of compilation
\usepackage[margin=1in]{geometry} % Required for setting margins
\parindent0pt % Set indentation to 0 pt
\usepackage[english]{babel} % English language/hyphenation

\declarecommand{\toggledraft}{
	\usepackage[firstpageonly=true]{draftwatermark}
	\def\answerborderwidth{1pt}
}

%------------------------------------------------------------------------
%	COLORS
%------------------------------------------------------------------------
\usepackage{xcolor} % required to defined colours
	\definecolor{DarkBlue}{HTML}{25355A}
	\definecolor{LightBlue}{HTML}{007FA3}
	\definecolor{Yellow}{HTML}{A37500}
	\definecolor{Red}{HTML}{A3002E}

%------------------------------------------------------------------------
%	FONTS
%------------------------------------------------------------------------
\usepackage[charter,cal=cmcal]{mathdesign} % Specify font for serif family
\usepackage{avant} % Specify avant font for sans serif family
\usepackage{mathtools} % Replaces amsmath
\usepackage{sectsty} % Specify section formatting
	\sectionfont{\sffamily\color{DarkBlue}}
	\subsectionfont{\sffamily\color{LightBlue}}
	\subsubsectionfont{\small\sffamily}

%------------------------------------------------------------------------
%	TABLES, IMAGES, AND FIGURES
%------------------------------------------------------------------------

\usepackage{array} % Required for tables
\usepackage{graphicx} % Required for including images
	\graphicspath{{img/}} % Specifies the directory where pictures are stored
\usepackage{tikz} % Required for creating plots
	\usetikzlibrary{decorations.pathreplacing}
\usepackage{tikz-3dplot} % Required for 3D plots
\usepackage{tikz-cd} % Required for commutative diagrams
\usepackage{pgfplots}
	\pgfplotsset{compat=1.12}
	\usepgfplotslibrary{colormaps}
	\usepgfplotslibrary{patchplots}
	\usepgfplotslibrary{fillbetween}

%------------------------------------------------------------------------
%	LINKS AND REFERENCES
%------------------------------------------------------------------------

\usepackage[hidelinks, urlcolor=Red, linkcolor=Yellow, colorlinks=true]{hyperref}   
\usepackage[noabbrev,capitalise,nameinlink]{cleveref}

%------------------------------------------------------------------------
%	BOXES, CHOICES, AND LISTS
%------------------------------------------------------------------------

\usepackage{enumerate}
\usepackage{enumitem} % Customize lists
% \setlist{nolistsep} % Reduce spacing between bullet points and numbered lists

\declarecommand\partlabel{(\thequestion\alph{partno})} % parts are labelled (1a)

\usepackage{environ}
%%% Answer environment makes a framed box of a chosen height
\NewEnviron{answer}[1]{
	\fullwidth{
	\vspace*{-10pt}
		\par\nobreak\vspace{\ht\strutbox}\noindent
		\setlength{\fboxrule}{\answerborderwidth}
		\fbox{% 
		\parbox[c][#1][t]{\dimexpr\linewidth-2\fboxsep}{
		\hrule width \hsize height 0pt \color{LightBlue} 
		\BODY
		}%
		}%
	}
}
\def\answerborderwidth{0pt} % Set answer boxes to be invisible by default

% Define remark environment
\NewEnviron{remark}{
	\fullwidth{
	\vspace*{-10pt}
		\color{Red} \textbf{Remark:}  
		\BODY
	}
}

\CorrectChoiceEmphasis{\color{LightBlue}\bf}	

\declarecommand{\CorrectChoiceBox}{
	\checkboxchar{$\Box$} 
	\checkedchar{$\blacksquare$}
}
\declarecommand{\CorrectChoiceCircle}{
	\checkboxchar{\tikz[baseline=-0.6ex]\draw[black,thick] (0,0) circle (1ex);}
	\checkedchar{\tikz[baseline=-0.6ex]\draw[LightBlue,fill=LightBlue,thick] (0,0) circle (1ex);}
}

 
%------------------------------------------------------------------------
%	THEOREMS
%------------------------------------------------------------------------

\usepackage{amsthm}

%%% Specify theorems
\newtheorem{theorem}{Theorem}
\newtheorem*{definition}{Definition}

\newtheoremstyle{named}{}{}{\itshape}{}{\bfseries}{.}{.5em}{\thmnote{#3}}
\theoremstyle{named}
\newtheorem*{namedtheorem}{Theorem}


%% Setup proof lines
\NewEnviron{lines}{
	\vspace*{-5pt}
	\begin{center}
	\fbox{\begin{minipage}{0.9\textwidth}
	\begin{enumerate}[label=\arabic*.]
		\it \BODY
	\end{enumerate}
	\end{minipage}
	}
	\end{center}
}
 
%------------------------------------------------------------------------
%	HEADER & FOOTER
%------------------------------------------------------------------------

%%% Define footnote without marker
\declarecommand{\blfootnote}[1]{%
  \begingroup
  \declarecommand\thefootnote{}\footnote{#1}%
  \addtocounter{footnote}{-1}%
  \endgroup
}

%%% Specify header and footer
\pagestyle{headandfoot} 	\runningfootrule  
	\runningfooter{\coursecode}{\docname\ - Page \the\numexpr\thepage\ of \numpages}{\duedate}

%------------------------------------------------------------------------
%	TITLE PAGE
%------------------------------------------------------------------------


%%% Define title
\declarecommand{\maketitle}{
	\begin{center}
		{\color{DarkBlue}\Large\sffamily\bfseries\title} \\[2pt]
		{\color{LightBlue}\Large\sffamily\bfseries\subtitle}
		\blfootnote{Created and revised by \author. Last updated \today \, at \currenttime.}\\
		{\sffamily\subsubtitle} 
	\end{center}
 
}

%%% Define title page
\declarecommand{\makeinstructions}
{ 
\section*{Instructions}
This problem set is based on \docinfo. Please read the \href{\urlFAQ}{Problem Set FAQ} for details on submission policies, collaboration  rules, and general instructions.  
\begin{itemize}
	\item \textbf{\href{\urlLMS/pages/tutorials}{Tutorials} on \tutorialdate\, will help you with \docname.} You will work with peers and get help from TAs. Before attending, seriously attempt these problems and prepare initial drafts.
	\item \textbf{Submissions are only accepted by \href{\urlGrading}{Gradescope}}. Do not send anything by email.  Late submissions are not accepted under any circumstance. Remember you can resubmit anytime before the deadline. 
	\item \textbf{Submit your polished solutions using only this template PDF.} You will submit a single PDF with your full written solutions. If your solution is not written using this template PDF (scanned print or digital) then you will receive zero. Do not submit rough work. Organize your work neatly in the space provided.  
	\item \textbf{Show your work and justify your steps} on every question, unless otherwise indicated. Put your final answer in the box provided, if necessary. 
\end{itemize}
We recommend you write draft solutions on separate pages and afterwards write your polished solutions here. You must fill out and sign the academic integrity statement below; otherwise, you will receive zero. 

\section*{Academic integrity statement}
%%% Student information
% Person A
\fbox{
\begin{minipage}{\textwidth}
{
	\vspace{0.2in}
	
	\makebox[\textwidth]{\sffamily Full Name:\enspace{\bfseries\large\FirstStudentName\,}\hrulefill}
	
	\vspace{0.2in}
	
	\makebox[\textwidth]{\sffamily Student number:\enspace{\bfseries\large\FirstStudentNumber\,}\hrulefill}
	
	\vspace{0.1in}
	
}
\end{minipage}
}

\vspace*{0.1in}

% Person B
\fbox{
\begin{minipage}{\textwidth}
{
\vspace{0.2in}

\makebox[\textwidth]{\sffamily Full Name:\enspace{\bfseries\large\SecondStudentName\,}\hrulefill}

\vspace{0.2in}

\makebox[\textwidth]{\sffamily Student number:\enspace{\bfseries\large\SecondStudentNumber\,}\hrulefill}

\vspace{0.1in}

}
\end{minipage}
}
~

I confirm that:

\begin{itemize} 
	\item I have read and followed the policies described in the \href{\urlFAQ}{Problem Set FAQ}. 
	\item I have read and understand the rules for collaboration on problem sets described in the Academic Integrity subsection of the syllabus. I have not violated these rules while writing this problem set. 
	\item I understand the consequences of violating the University's academic integrity policies as outlined in the \href{http://www.governingcouncil.utoronto.ca/policies/behaveac.htm}{Code of Behaviour on Academic Matters}. I have not violated them while writing this assessment.
\end{itemize}
By signing this document, I agree that the statements above are true. 

% You should sign this PDF after compiling. Do not write your signature using LaTeX.
\vspace{0.2in}
{\large 
	\makebox[\textwidth]{\sffamily Signatures: 1)\enspace\hrulefill} 
	
	\vspace{0.2in}
	
	\makebox[\textwidth]{\sffamily \hspace*{22mm} 2)\enspace\hrulefill} 
}
\vfill
\pagebreak
}



%%% Uncomment if you are writing a draft
%\toggledraft

%%% Uncomment if you want to show your multiple choice answers
\printanswers

%%% Local document information
\declarecommand{\docname}{Problem Set 6}		\usepackage{neuralnetwork}
\declarecommand{\dueday}{Friday}
\declarecommand{\duedate}{February 10, 2023}
\declarecommand{\duetime}{13:00 ET}
\declarecommand{\tutorialdate}{Tuesday February 7, 2023}
\declarecommand{\docinfo}{\href{\urlFAQ/modules}{Module G: Integrals} (G7) and \href{\urlFAQ/modules}{Module H: Integral applications} (H1 to H3) and \href{\urlFAQ/modules}{Module I: Integration methods} (I1 and I2)}

%%% Uncomment both if you are a MAT237 teaching member writing the official  solutions
%\declarecommand{\makeinstructions}{\vfill\pagebreak}
%\declarecommand{\docname}{Problem Set 5 solutions}
%\declarecommand{\authors}{C. Davies, A. Qiu, and A. Zaman}

%%% Student information
\declarecommand{\FirstStudentName}{} 
\declarecommand{\FirstStudentNumber}{}

\declarecommand{\SecondStudentName}{} % leave empty if you are submitting individually
\declarecommand{\SecondStudentNumber}{} % leave empty if you are submitting individually

%%% Custom math commands
%% Standard
\declarecommand{\ds}{\displaystyle} % force math styling
\declarecommand{\emptyset}{\varnothing} % better empty set
\declarecommand{\epsilon}{\varepsilon} % better epsilon
\declarecommand{\R}{\mathbb{R}} % reals
\declarecommand{\Z}{\mathbb{Z}} % integers
\declarecommand{\Q}{\mathbb{Q}} % rationals
\declarecommand{\N}{\mathbb{N}} % naturals
\declarecommand{\C}{\mathbb{C}} % complex
\DeclarePairedDelimiter\abs{\lvert}{\rvert} % absolue value
\DeclarePairedDelimiter\norm{\lVert}{\rVert} % norm

%% Linear algebra
\declarecommand{\mat}[1]{\begin{bmatrix*}[r]#1\end{bmatrix*}}
\declarecommand{\matc}[1]{\begin{bmatrix}#1\end{bmatrix}} % matrix
\DeclareMathOperator{\Span}{span} % span
\DeclareMathOperator{\Img}{img} % image
\DeclareMathOperator{\Ker}{ker} % kernel
\DeclareMathOperator{\Id}{id} % identity map
\DeclareMathOperator{\Range}{range} % range
\DeclareMathOperator{\Rref}{rref} % row reduced echelon form
\DeclareMathOperator{\Rank}{rank} % rank
\DeclareMathOperator{\Null}{null} % null space
\DeclareMathOperator{\Nullity}{nullity} % nullity
\DeclareMathOperator{\Proj}{proj} % projection
\DeclareMathOperator{\Dim}{dim} % dimension

%% Topology
\DeclareMathOperator{\Int}{int}
\DeclareMathOperator{\Cl}{cl}

%% Calculus
\DeclareMathOperator{\Length}{length} % length
\DeclareMathOperator{\Area}{area} % area
\DeclareMathOperator{\Vol}{vol} % volume
\DeclareMathOperator{\Grad}{grad} % gradient
\DeclareMathOperator{\Curl}{curl} % curl
\DeclareMathOperator{\Div}{div} % divergence

\begin{document}



\maketitle

\vspace*{-15pt}
\makeinstructions
 
%%% PROBLEMS
\section*{Problems} 

\CorrectChoiceCircle % for multiple choice questions

%%% Do not change the height of this box; your work must fit inside
%\begin{answer}{450pt}
%% Insert your answer here
%\end{answer}

\begin{questions}


%%% QUESTION 1
\question  No justification is necessary for any part of this question. 

\begin{parts}

\item Find a bounded set $S \subseteq \R^n$ such that $f$ is integrable on $S$ for all bounded functions $f : \R^n \to \R$. 
%% Do not change the height of this box; your work must fit inside
\begin{answer}{80pt}
% Insert your answer here
Let $S = \{0\} \subseteq \R^n$.
\end{answer}
	
\item Find a bounded function $f : \R^n \to \R$ such that $f$ is integrable on $S$ for all bounded sets $S \subseteq \R^n$.  
%% Do not change the height of this box; your work must fit inside
\begin{answer}{80pt}
% Insert your answer here
Let $f(x) = 0, \forall x \in \R^n$.
\end{answer}


\item Find a bounded set $S \subseteq \R^n$ and a bounded function $f : \R^n \to \R$ such that 
\begin{itemize}
	\item $f$ is not integrable on $S$.
	\item $f$ is integrable on $\partial S$, is integrable on $\overline{S}$, and is integrable on $S^o$. 
\end{itemize}
%% Do not change the height of this box; your work must fit inside
\begin{answer}{120pt}
% Insert your answer here
Let $S = (\Q \cap [0,1])^n$. Let $f(x) = 1$, $\forall x \in \R^n$.
\end{answer}


\item Find a bounded set $S \subseteq \R^n$ and a bounded function $f : \R^n \to \R$ such that 
\begin{itemize}
	\item $f$ is integrable on $S$, and is integrable on $S^o$.
	\item $f$ is not integrable on $\partial S$, and is not integrable on $\overline{S}$. 
\end{itemize}
%% Do not change the height of this box; your work must fit inside
\begin{answer}{120pt}
% Insert your answer here
Let $S = \Q^n \cap [0,1]^n$. Let $f(x) = 0$ for $x \in \Q^n$, and $f(x) = 1$ for $x \in \R^n \setminus \Q^n$.
\end{answer}

\end{parts}



\pagebreak
%%% QUESTION 2
\question For each part, decide which statement is true by filling EXACTLY ONE circle. No justification is necessary. 


Let $S \subseteq \R^n$ be a Jordan measurable set. Let $f : \R^n \to \R$ be bounded. Assume $f$ is integrable on $S$. 



\CorrectChoiceCircle
\begin{parts}
	
	\item Is $f$ integrable on the boundary $\partial S$?	\\[-3pt]
	
	% Replace \choice with \CorrectChoice to select an option
	\begin{checkboxes}
		\choice We cannot determine whether $f$ is integrable on $\partial S$ without more information. \\[-3pt]
		\choice $f$ is not integrable on $\partial S$.\\[-3pt]
		\choice $f$ is integrable on $\partial S$ but we cannot determine the value of $\ds\int_{\partial S} f dV$ without more information. 	
		\CorrectChoice $f$ is integrable on $\partial S$ and $\ds\int_{\partial S} f dV =0$.
		\choice $f$ is integrable on $\partial S$ and $\ds\int_{\partial S} f dV = \ds\int_S fdV$.
	\end{checkboxes}
	\medskip

	% Replace \choice with \CorrectChoice to select an option
	\item Is $f$ integrable on the interior $S^o$?	\\[-3pt]
	\begin{checkboxes}
		\choice We cannot determine whether $f$ is integrable on $S^o$ without more information. \\[-3pt]
		\choice $f$ is not integrable on $S^o$.\\[-3pt]
		\choice $f$ is integrable on $S^o$ but we cannot determine the value of $\ds\int_{S^o} f dV$ without more information. 	
		\choice $f$ is integrable on $S^o$ and $\ds\int_{S^o} f dV =0$.
		\CorrectChoice $f$ is integrable on $S^o$ and $\ds\int_{S^o} f dV = \ds\int_S fdV$.
	\end{checkboxes}
	\medskip

	% Replace \choice with \CorrectChoice to select an option
	\item Is $f$ integrable on the closure $\overline{S}$?\\[-3pt]
	\begin{checkboxes}
		\choice We cannot determine whether $f$ is integrable on $\overline{S}$ without more information.  \\[-3pt]
		\choice $f$ is not integrable on $\overline{S}$.\\[-3pt]
		\choice $f$ is integrable on $\overline{S}$ but we cannot determine the value of $\ds\int_{\overline{S}} f dV$ without more information. 	
		\choice $f$ is integrable on $\overline{S}$ and $\ds\int_{\overline{S}} f dV =0$.
		\CorrectChoice $f$ is integrable on $\overline{S}$ and $\ds\int_{\overline{S}} f dV = \ds\int_S fdV$.
	\end{checkboxes}
	\medskip

	% Replace \choice with \CorrectChoice to select an option	
	\item Is $f$ integrable on the complement $S^c$?\\[-3pt]
	\begin{checkboxes}
		\CorrectChoice We cannot determine whether $f$ is integrable on $S^c$ without more information.  \\[-3pt]
		\choice $f$ is not integrable on $S^c$.\\[-3pt]
		\choice $f$ is integrable on $S^c$ but we cannot determine the value of $\ds\int_{S^c} f dV$ without more information. 	
		\choice $f$ is integrable on $S^c$ and $\ds\int_{S^c} f dV = 0$.
		\choice $f$ is integrable on $S^c$ and $\ds\int_{S^c} f dV = - \ds\int_S fdV$.
	\end{checkboxes}
	
\end{parts}

 

\pagebreak
%%% QUESTION 3
\question \label{CentreMass} 
For a system of point masses at positions $x_1,\dots,x_N \in \R^n$ with masses $m_1,\dots,m_N > 0$ respectively, the centre of mass of this system is given by 
\begin{equation}
\frac{\sum_{i=1}^N m_i x_i }{\sum_{i=1}^N m_i} \in \R^n. 
	\tag{$\star$}
	\label{Principle}
\end{equation}
You may take this physical principle for granted. 


Let $\delta :R \to [0,\infty)$ be the density of a rectangular object $R \subseteq \R^n$ with positive mass. Assume $\delta$ is continuous on $R$ aside from a set of zero Jordan measure. SpongeBob attempts to derive an integral formula for the centre of mass, but skips some crucial explanation and makes a mathematical error. 

\begin{lines}
	\item Let $P$ be a partition of $R$ with subrectangles $\{R_i\}_{i \in I}$.
	\item For each $i \in I$, let $x_i^* \in R_i$ be a sample point. 
	\item By \eqref{Principle}, the centre of mass of $R$ is approximately equal to $\dfrac{\sum_{i \in I} x_i^* \delta(x_i^*) \Vol(R_i)  }{ \sum_{i\in I} \delta(x_i^*) \Vol(R_i)} \in \R^n$. 
	\item Taking $|I| \to \infty$, the centre of mass is therefore equal to $\dfrac{\int_R x \delta(x) dV }{\int_R  \delta dV} \in \R^n$.
\end{lines}
\begin{parts}
	\item  Line 3 applies \eqref{Principle} without enough detail. What objects are treated as point masses? How are masses approximated? Under what mathematical or physical assumptions can these two heuristics be considered reasonable? Explain in 50 to 100 words. Use full sentences, be concise, and use symbols sparingly.
	
	%% Do not change the height of this box; your work must fit inside
\begin{answer}{260pt}
% Insert your answer here
The subrectangles are treated as point masses. For each subrectangle, its mass is approximated by a sample point lying inside it. These are reasonable with the following assumptions: if we assume the subrectangles are arbitrarily small, then their volumes are roughly uniform and approaching that of a point, which is zero. If we assume the density is continuous, then at arbitrarily small regions, it is roughly constant, so the density at a sample point is roughly equal to the density at any point in a given subrectangle.
\end{answer}
	
	\item Line 4 takes an incorrect limit. What limit should be taken instead? Also, state which theorem could be used to justify that this corrected limit is equal to the corresponding expression. Do not justify it. 

	%% Do not change the height of this box; your work must fit inside
\begin{answer}{60pt}
% Insert your answer here
For line 1, construct a sequence of partitions $P_1, P_2, P_3, ...$ of R such that $|| P_N || \to 0$ as $N \to \infty$. Thus, in line 4, take the limit $N \to \infty$. The theorem is 7.3.18.
\end{answer}


\end{parts}




\pagebreak
%%% QUESTION 4
\question  Define $\Omega = [-2023,2023]^4$. Let $(\Omega,\Sigma,\mathbb{P})$ be a continuous probability space in $\R^4$ with probability density function $\phi$. 
Choose $(A,B,C,D) \in \Omega$ randomly according to your continuous probability space. You will prove that the $2 \times 2$ matrix
\[
M = \matc{ A & B \\ C & D} 
\]
is invertible with probability one. 
\begin{parts}

\item Define the set
\[
S = \{ (a,b,c,d) \in \Omega : ad - bc = 0\}.
\]
Assuming $S$ is an event in $\Sigma$ and $\mathbb{P}(S) = 0$, explain why $M$ will be invertible with probability one. 

%% Do not change the height of this box; your work must fit inside
\begin{answer}{330pt}
% Insert your answer here
Assume $\mathbb{P}(\Omega) = 1$. $\Omega$ is Jordan measurable since it is a rectangle. Since $S \in \Sigma$, $\Omega \setminus S \in \Sigma$ by Theorem 8.4.3b. Since $S$ and $\Omega \setminus S$ are disjoint, $\mathbb{P}(\Omega) = \mathbb{P}(S \cup \Omega \setminus S) = \mathbb{P}(S) + \mathbb{P}(\Omega \setminus S)$ by Theorem 8.4.7c. $\mathbb{P}(\Omega) = 1$ and $\mathbb{P}(S) = 0$ by assumption, so $\mathbb{P}(\Omega \setminus S) = 1$.\

\

$\Omega \setminus S$ is the event that $S$ does not happen, i.e., that $M$ is invertible. Thus, $M$ is invertible with probability 1.
\end{answer}

\item Prove that if $S$ has zero Jordan measure, then $S \in \Sigma$ and $\mathbb{P}(S) = 0$. 

%% Do not change the height of this box; your work must fit inside
\begin{answer}{150pt}
By Lemma 7.6.8a, since $S$ has zero Jordan measure, it is Jordan measurable. Together with the fact that $S \subseteq \Omega$, by definition of $\Sigma$, this means $S \in \Sigma$.\newline\

$S \subseteq \Omega$ is bounded, and $\phi$ is bounded by construction. Since $S$ has zero Jordan measure, by Theorem 7.7.8a and the definition of the probability function $\mathbb{P}$, $\mathbb{P}(S) = \int_S \phi dV = 0$.
\end{answer}


\item Fill the remaining gap in the proof by showing $S$ has zero Jordan measure. \textit{Hint:} Proceed by definition. Cover the origin in $\R^4$ with a small rectangle depending on $\epsilon$ and then use Sard's theorem to cover the other pieces which depend on $\epsilon$.

%% Do not change the height of this box; your work must fit inside
\begin{answer}{590pt}
% Insert your answer here
Let $\epsilon > 0$. WLOG, assume $\epsilon < 1$. Take $r = \sqrt[4]{\epsilon/32}$. Let $(a,b,c,d) \in S$.\newline\

Either $(a,b,c,d) \in (-r,r)^4$ or $(a,b,c,d) \notin (-r,r)^4$. For the former case, notice $(-r,r)^4 \subseteq [-r,r]^4$ and vol$([-r,r]^4) = (\sqrt[4]{\epsilon/2})^4 = \epsilon/2$.\newline\

Consider the following functions:\

Define $f_1: \R \setminus \{0\} \times \R \times \R \to \R^4$ by $f(x,y,z) = (x,y,z,yz/x)$ for $y, z \in \R, x \in \R \setminus \{0\}$.\

Define $f_2: \R \times \R \setminus \{0\} \times \R \to \R^4$ by $f(x,y,w) = (x,y,xw/y,w)$ for $x, w \in \R, y \in  \R \setminus \{0\}$.\

Define $f_3: \R \times \R \setminus \{0\} \times \R \to \R^4$ by $f(x,z,w) = (x,xw/z,z,w)$ for $x, w \in \R, z \in  \R \setminus \{0\}$.\

Define $f_4: \R \times \R \times \R \setminus \{0\} \to \R^4$ by $f(y,z,w) = (yz/w,y,z,w)$ for $y, z \in \R, w \in  \R \setminus \{0\}$.\

Note the above functions are all $C^1$ on their respective domains.\newline\

For $(a,b,c,d) \notin (-r,r)^4$, consider the following subcases:\newline\

Subcase $a \leq -r$: $(a,b,c,d) \in f_1([-2023,-r]\times[-2023,2023]^2)$.\

Subcase $a \geq r$: $(a,b,c,d) \in f_1([r,2023]\times[-2023,2023]^2)$.\

Subcase $b \leq -r$: $(a,b,c,d) \in f_2([-2023,2023]\times[-2023,-r]\times[-2023,2023])$.\

Subcase $b \geq r$: $(a,b,c,d) \in f_2([-2023,2023]\times[r,2023]\times[-2023,2023])$.\newline\

The subcases $c \leq -r$, $c \geq r$, $d \leq -r$, $d \geq r$ are similar, using the images of $f_3, f_3, f_4$, and $f_4$ respectively.\newline\

By Sard's theorem (Theorem 7.5.12), since $f_1, f_2, f_3$, and $f_4$ are $C^1$ on their open domains, their images as specified in the subcases have zero Jordan measure in $R^4$.\newline\

Notice that $S$ is contained in the union of $(-r,r)^2$ and the eight images of $f_1, f_2, f_3$, and $f_4$ as specified in the subcases. Since vol$((-r,r)^4) \leq \epsilon/2$ and the images have zero volume, this union has volume $< \epsilon$. By definition 7.5.1, $S$ has zero Jordan measure.

\end{answer}

\end{parts}


\pagebreak

%%% QUESTION 5
\question \label{Fubini} Let $S = \{ (x,y) \in \R^2 : -2 \leq x \leq 2, 0 \leq y \leq \sqrt{4-x^2} \}$ and define $f : S \to \R$ by $f(x,y) = 3+\sqrt{4-x^2-y^2}$.
\begin{parts}

\item Prove that $f$ is integrable on $S$.
%% Do not change the height of this box; your work must fit inside
\begin{answer}{580pt}
% Insert your answer here
$S \in B_3(0)$ so it is bounded. Define $\phi : \R \to \R^2$ by $\phi(x) = (2 \cos x, 2 \sin x)$ for $x \in \R$. Define $h: \R \to \R^2$ by $h(t) = (t, 0)$ for $t \in R$. Notice $\partial S = \{ (x,y) \in \R^2 : x^2 + y^2 = 2, y \geq 0 \} = \phi([ 0, \pi ]) \cup h([-2, 2])$. Since $\phi$ and $h$ are $C^1$, by Sard's theorem, $\partial S$ has zero Jordan measure. Hence, by definition 7.6.3, $S$ is Jordan measurable.\newline\

Note $f$ is bounded on $S$ since $\forall x \in S, 3 \leq f(x) \leq 5$. The set of discontinuities of $f$ in $S$ is empty and hence has zero Jordan measure. By theorem 7.7.4, $f$ is integrable on $S$.
\end{answer}
	
\item Formally verify the assumptions of Fubini's theorem to prove that 
\[
\iint_S f dA = \int_{-2}^2 \int_0^{\sqrt{4-x^2}} 3+\sqrt{4-x^2-y^2} \, dy \, dx.  
\]	
Do not calculate any integrals. 
%% Do not change the height of this box; your work must fit inside
\begin{answer}{570pt}
% Insert your answer here

By part a, $f$ is integrable on $S$ and $\iint_{S} f dA$ exists. Define $Q = [-2, 2] \times [0, 2] \supseteq S$. Define $g: \R^2 \to \R$ by $g(x, y) = \chi_S(x, y) f(x, y)$ for $(x, y) \in \R^2$. Since $\iint_{S} f dA = \iint_{Q} g dA$ by definition 7.7.1, $g$ is integrable on $Q$. \newline\

Let $x \in [-2, 2]$. Define $g^x : [0, 2] \to \R$ by $g^x (y) = g(x, y)$ for $y \in [0, 2]$.\newline\

Note $g^x(y) = \chi_S(x, y) f(x, y) = \chi_{[0, \sqrt{4-x^2}]}(y) f(x, y)$ by definition of $S$. Thus, for all $x \in [-2, 2]$, the set of discontinuities of $g^x$ on $[0, 2]$ is $\{(x, \sqrt{4 - x^2}) \in Q : x \in [-2, 2]\} = \phi([0, \pi])$, for $\phi$ as defined in part a. By Sard's theorem (Theorem 7.5.12), since $\phi$ is $C^1$ on $\R$, this set has zero volume. By Theorem 7.7.17, $g^x$ is integrable on $[0, 2]$, $\forall x \in [-2, 2]$.\newline\

By Fubini's theorem (Theorem 9.7.17), the iterated integral $dydx$ exists and $\iint_S f dA = \iint_{Q} g dA = \int_{-2}^2 \int_{0}^2 g(x,y) dydx = \int_{-2}^2 \int_0^{2} \chi_{[0, \sqrt{4-x^2}]}(y) f(x,y) dydx = \int_{-2}^2 \int_0^{\sqrt{4-x^2}} f(x,y) dydx$.

\end{answer}

\end{parts}

 
\pagebreak

%%% QUESTION 6
\question Consider again the integral
\[
I = \int_{-2}^2 \int_0^{\sqrt{4-x^2}} 3+\sqrt{4-x^2-y^2} \, dy \, dx
\]
from Question \ref{Fubini}. You could evaluate $I$ by direct calculation, but instead you will try two different ways. 
\begin{parts}

\item Interpret $I$ as a volume of a solid $T \subseteq \R^3$. Define the solid $T$ using set builder notation, and include sketches in $\R^3$ to support your argument. Compute the volume using geometry of known solids in $\R^3$. Do not evaluate any integrals. 
%% Do not change the height of this box; your work must fit inside
\begin{answer}{530pt}
% Insert your answer here
\end{answer}
\pagebreak
\item Sketch the region of integration using the $dx \, dy$ order along with a typical slice. Label your figure. 
%% Do not change the height of this box; your work must fit inside
\begin{answer}{90pt}
% Insert your answer here
\end{answer}

\item  Assuming Fubini's theorem applies without justification, evaluate $I$ using the $dx \, dy$ order of integration. Show your steps when calculating single variable integrals. Do not use polar coordinates. 
%% Do not change the height of this box; your work must fit inside
\begin{answer}{480pt}
% Insert your answer here
\end{answer}

\end{parts}

\end{questions}

\end{document}