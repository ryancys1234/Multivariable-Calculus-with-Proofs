\documentclass[answers]{exam}
\usepackage{amsfonts, amsmath}

\begin{document}

\begin{center}
{\Large MAT237Y1 - Interview Activity}
\end{center}

\begin{questions}

\question \textit{Claim:} Let $S\subseteq\mathbb{R}^n$ be a finite set. Prove that $S$ has zero Jordan measure.

\begin{solution}
Fix $\varepsilon > 0$. Set $N = |S|$, the cardinality of the set. Label the elements of $S$ as $x_1,x_2,...,x_N$. For $i\in\{1,...,N\}$, define the rectangles\begin{align*}
    R_i = \left[(x_i)_1-\sqrt[n]{\frac{\varepsilon}{2^{n+1} N}},\ (x_i)_1+\sqrt[n]{\frac{\varepsilon}{2^{n+1} N}}\ \right]\times...\times\left[(x_i)_n-\sqrt[n]{\frac{\varepsilon}{2^{n+1} N}},\ (x_i)_n+\sqrt[n]{\frac{\varepsilon}{2^{n+1} N}}\ \right]\ .
\end{align*} Notice that $x\in R_i$ by the construction of $R_i$ and\begin{align*}
    \text{vol}(R_i)=(2\sqrt[n]{\frac{\varepsilon}{2^{n+1} N}})^n=(\sqrt[n]{\frac{\varepsilon}{2N}})^n=\frac{\varepsilon}{2N}\ .
\end{align*}Therefore, $S\subseteq\bigcup_{i=0}^N R_i$ and\begin{align*}
    \sum_{i=1}^N \text{vol}(R_i)=N\frac{\varepsilon}{2N}=\frac{\varepsilon}{2}<\varepsilon
\end{align*}as needed.
\end{solution}

\end{questions}

The skills needed to produce this solution include:\begin{itemize}
    \item The ability to recall the definition of a set having zero Jordan measure, which in turn involves the definitions of rectangles, the volume of a rectangle, and the finite union and sum operations.
    \item The ability to match the structure of the solution with the format of the definition. The steps must be taken in this order for correctness: fix an arbitrary $\varepsilon$, set an appropriate $N$, construct specific rectangles depending on $\varepsilon$ and $N$, then ensure the conditions of the definition are satisfied.
    \item The ability to recognize what a finite set is and how to construct a simple rectangle that uses the properties of such a set to meet the conditions of the definition.
    \item The ability to recognize and avoid potential sources of confusion. Above $(x_i)_1$ was used to avoid confusion with notation such as $x_{i1},...,x_{in}$, where it would be difficult to distinguish between the different subscripts.
\end{itemize}

\end{document}